\documentclass[margin,line]{res}
\usepackage{amsmath, amssymb}
\usepackage{eurosym}
\usepackage{hyperref}

\oddsidemargin -.5in
\evensidemargin -.5in
\textwidth=6.0in
\itemsep=0in
\parsep=0in

\newenvironment{list1}{
  \begin{list}{\ding{113}}{%
      \setlength{\itemsep}{0in}
      \setlength{\parsep}{0in} \setlength{\parskip}{0in}
      \setlength{\topsep}{0in} \setlength{\partopsep}{0in} 
      \setlength{\leftmargin}{0.17in}}}{\end{list}}
\newenvironment{list2}{
  \begin{list}{$\bullet$}{%
      \setlength{\itemsep}{0in}
      \setlength{\parsep}{0in} \setlength{\parskip}{0in}
      \setlength{\topsep}{0in} \setlength{\partopsep}{0in} 
      \setlength{\leftmargin}{0.2in}}}{\end{list}}


\begin{document}

\name{Ilya Mandel \vspace*{.1in}}

\begin{resume}
\section{\sc Contact Information}
\vspace{.05in}
\begin{tabular}{@{}p{4in}p{2in}}
School of Physics and Astronomy, Monash University \\
19 Rainforest Walk, Clayton,  VIC, 3168, Australia \\
{\it E-mail:}  ilya.mandel@monash.edu \\
{\it WWW:} http://ilyamandel.github.io \\         
\end{tabular}


%\section{\sc Research Interests}
%\begin{tabular}{@{}p{3in}p{3in}}
%Gravitational-Wave Astronomy & Evolution of Massive Binaries \\
%Astrophysics of Compact Objects & Astrostatistics \\  %Gravitational-Wave Data Analysis
%\end{tabular}

\section{\sc Education}
{\bf California Institute of Technology}, Pasadena, CA USA\\
\vspace*{-.1in}
\begin{list1}
\item[] Ph.D. in Physics (conferred June 2008)
\begin{list2}
\vspace*{.05in}
\item Dissertation Title:  ``The Three Ss of Gravitational Wave Astronomy: 
Sources, Signals, Searches'' 
\item Advisor:  Kip S. Thorne
\end{list2}
\vspace*{.05in}
\item[] M.S. in Physics, 2003 
\end{list1}

{\bf Stanford University}, Stanford, CA USA\\
\vspace*{-.1in}
\begin{list1}
\item[] M.S. in Computer Science (Theory specialization), 2001
\vspace*{.05in}
\item[] B.S. in Physics, with Distinction and Departmental Honors, 2000 
\end{list1}


%\section{\sc Work Experience}
\section{\sc Academic Experience}

{\bf Monash University, Australia}\\
{\em Professor of Astrophysics}  \hfill {\bf 2019 - present}

{\bf University of Birmingham, UK}\\
{\em Honorary Professor}  \hfill {\bf 2019 - present}\\
{\em Professor of Theoretical Astrophysics} \hfill {\bf 2016 - 2019}\\
{\em Senior Lecturer} \hfill {\bf 2014 - 2016}\\
{\em Lecturer} \hfill {\bf 2011 - 2014}

{\bf NSF Astronomy \& Astrophysics Postdoctoral Fellow \hfill {2009 - 2011}}\\
At Northwestern 9/2009-6/2010; at MIT 7/2010-8/2011.
 
{\bf Northwestern University}, Evanston, IL USA\\
\vspace*{.02in}
{\em Postdoctoral Scholar in Theoretical Astrophysics  \hfill {\bf 2007 - 2009}}\\
Mentor: Vicky Kalogera
%Research on astrophysical sources of gravitational waves and data-analysis challenges posed by current and planned gravitational-wave detectors

{\bf California Institute of Technology}, Pasadena, CA USA\\
\vspace*{.02in}
{\em Research Assistant} \hfill {\bf 2006 - 2007}\\
\vspace*{0.02in}
 {\em Teaching Assistant} \hfill {\bf 2001 - 2006}
%Teaching recitation sections for sophomore physics (waves, quantum mechanics, statistical mechanics); supervision of classical and modern physics lab courses and computational methods in physics

{\bf Stanford University}, Stanford, CA USA\\
\vspace*{.02in}
{\em Research Assistant, Gravity Probe B} \hfill {\bf 1996 - 2003}\\
%Physics research in the Data Analysis group of GP-B: data-analysis algorithm development, modeling, theoretical investigation of trapped magnetic flux
\vspace*{.02in}
{\em Research and Teaching Assistant, Computer Science} \hfill {\bf 1999}
%Research on the efficiency and feasibility of various routing algorithms in a congested network;  Teaching Assistant for ``Design and Analysis of Algorithms'' course

%{\bf Software Development}, Silicon Valley, CA USA\\ 
%{\em Software developer, Contractor} \hfill {\bf 1999-2002}\\
%Development in Java, including servlets and JSP, SQL, C/C++; algorithm design; software architecture %on contract for several companies, including Kana Communications, Epitrope, and TopCoder, Inc.

\section{\sc Awards} 
Australian Research Council Future Fellow, 2019-2023\\
Elected Patron of Birmingham AstroSoc student society, 2017\\
Breakthrough Prize in Fundamental Physics (part of LIGO team), May 2016\\
SAMSI (Statistical and Applied Mathematical Science Institute) Research Fellow, 2016--2017\\
College of Engineering and Physical Sciences paper of the month award, February 2016\\
Classical and Quantum Gravity highlights -- most cited article in 2 years (``Predictions for the rates of compact binary coalescences observable by ground-based gravitational-wave detectors''), 2012\\
Head of School commendation letters for excellence in teaching (multiple years, 2013--2018)\\
Kevin Westfold Distinguished Visitor, Monash Center for Astrophysics, 2014\\
NSF Astronomy and Astrophysics Postdoctoral Fellowship\\
Dean's Award for Academic Achievement (excellence in research), Stanford University\\
Tau Beta Pi (engineering honor society) Member\\
National Merit Scholar, National Advanced Placement Scholar\\
%Robert Byrd Scholarship Recipient\\


\section{\sc Service}

{\it International/national leadership:}\\
Member, Australian Research Council College of Experts (2022 -- current)\\
Aspen Center for Physics board member (2016 -- current)\\
Gravitational-Wave International Committee 3G Science Case team member, Binaries working group co-chair (2017 -- 2019)\\
Vice President, IAU Binary and Multiple Star Systems Commission (2022 -- present); Organising Committee Member (2018 -- present)\\
International Society on General Relativity \& Gravitation, Nominating Committee member (2016 -- current)\\
Chair, steering committee of the Australian National Institute for Theoretical Astrophysics (ANITA; 2020 -- current [deputy chair, 2019-2020])\\
Councillor, Astronomical Society of Australia (2019 -- 2021)\\
Council member, International Astrostatistics Association (2021 -- 2023)\\
International Statistics Institute (ISI) Astrostatistics Special Interest Group Management Committee (2022 -- current)\\
Science Meets Parliament representative, 2021\\

{\it Journal editing and review, grant review:}\\
Scientific Editor, AAS (Astrophysical Journal, ApJ Letters)\\
Editorial Board member, Nature Scientific Data\\
Reviewer for the Astrophysical Journal, Physical Review, Classical and Quantum Gravity, Monthly Notices of the Royal Astronomical Society, Philosophical Transactions of the Royal Society, Nature journals, Science (2008--current); Institute of Physics Trusted Reviewer (2020) \\
ARC College of Experts member (2022--2024)\\
Grant reviewer for NASA (USA; including as member of NASA Astrophysics Theory Grant Panel), NSF (USA), Hungarian Scientific Research Fund, STFC (UK), NWO (Netherlands Organisation for Scientific Research), European Research Council, CONICYT (Chile), ARC (Australia), CRC (Canada), Marsden grants (New Zealand) (2008--current)\\

{\it Collaboration leadership/membership:}\\
LIGO Scientific Collaboration member (2005--2016), LSC Council member (2012 -- 2016)\\
LSC internal reviewer (2009 -- 2016), intermediate mass black hole source group co-chair (2012 -- 2016)\\
LISA Consortium Member\\
Mock LISA Data Challenge and Parameter Estimation Taskforce member\\
Einstein Telescope Science Working Group member\\
LOFT science team member\\
Member of several electromagnetic transient surveys (e.g., ePESSTO) and multi-messenger  consortia for following up gravitational-wave signals (e.g., ENGRAVE, where I co-chair the Populations theory group)\\

{\it Meeting organisation:}\\
SOC member, inaugural South African GW workshop (2011)\\
Convener of relativistic astrophysics session, GR-20 (2013)\\
Co-organizer and SOC co-chair, ``Science with the first gravitational-wave detections'' (2013)\\
Lead organizer, Aspen Center for Physics Workshop on Ultra-compact Binaries as Laboratories for Fundamental Physics (2014)\\
Main organizer and SOC chair, Gravitational-wave Astrophysics meeting at COSPAR (2014)\\
Program Leader, SAMSI Program on Statistical, Mathematical and Computational Methods for Astronomy (2016-2017)\\
Main organizer and SOC chair, Massive Binary Evolution meeting at COSPAR (2016 [cancelled because of current events in Turkey])\\
Organizer, SAMSI workshop on Astrophysical Population Emulation and Uncertainty Quantification (2017)\\
SOC member, Alpine Cosmology Workshop (2017 -- current)\\
Co-director, Kavli Summer Program in Astrophysics 2017: Astrophysics with gravitational wave detections (2017)\\
Co-organizer and SOC co-chair, Lorentz Center Workshop ``And then there was light:  
Electromagnetic signatures of stellar mass binary black hole mergers'' (2017)\\
Co-organizer, Munich Institute for Astro- and Particle Physics program ``Precision gravity: from the LHC to LISA'' (2019)\\
SOC Member, EAS Symposium ``What have we learned from the observed population of gravitational wave sources?'' (2020)\\
%SOC Member, COSPAR event ``X- and Gamma-ray Counterparts of New Transients in the Multimessenger Era'' (2020/2021)\\
Astronomical Society of Australia 2020 (virtual) science meeting organizer\\
ANITA 2021 workshop (virtual) lead organiser\\
SOC co-chair, EAS session ``Where are the BH-NS binaries?'' (2021)\\
SOC member, GWPAW 2021\\
SOC member, Nuclear burning in massive stars: towards the formation of binary black holes (2021)\\
SOC member, Designing the Next-Generation EHT to Transform Black Hole Science (2021)\\
Lead organizer and SOC chair, Gravitational-Wave Physics and Astronomy Workshop, Melbourne (2022)\\

{\it Institutional contributions:}\\
University of Birmingham astrophysics seminar organizer (2012--2015)\\
University of Birmingham physics and astronomy colloquium organizer (2015--current)\\
University of Birmingham head of graduate admissions for astronomy (2012 -- current)\\
University of Birmingham Faculty Senate representative elected for Engineering and Physical Sciences (2016 -- current)\\
School and College Equality and Diversity Committee member, University of Birmingham (2014 -- 2017) \\
Monash University astrophysics seminar and school colloquium co-organiser (2019)\\




\section{\sc Mentoring}
{\it Postdoctoral associates:}\\
Trevor Sidery (2011 -- 2013, co-mentored with Alberto Vecchio), gravitational-wave data analysis\\
Walter del Pozzo (2013 -- 2016), tests of general relativity with compact binaries; now faculty at University of Pisa\\
Christopher Berry (2013 -- 2017, co-mentored with Alberto Vecchio), astrostatistics; now faculty at University of Glasgow via research faculty at Northwestern\\
Dorottya Sz\'{e}csi (2017 -- 2018), stellar evolution; now faculty in Nicolaus Copernicus University via Humboldt fellowship in Cologne\\
Tyrone Woods (2018), stellar and binary evolution; now Plaskett Fellow at the NRC Herzberg Astronomy and Astrophysics, Victoria, Canada\\
Ryosuke Hirai (2019 -- current), hydrodynamical modelling\\
Evgeni Grishin (2021 -- current), dynamics\\
Isobel Romero-Shaw (2021 -- 2022), X-ray binary population modelling\\
\\
{\it Post-graduate students:}\\
Vivien Raymond (2007 -- 2012, co-advised with Vicky Kalogera), parameter estimation on spinning binaries; now faculty at Cardiff University via Albert Einstein Institute postdoc and Caltech postdoctoral prize fellowship\\
Ben Farr (2009 -- 2014, co-advised with V.~Kalogera), MCMC techniques in LIGO data analysis; now faculty at University of Oregon via McCormick postdoctoral fellow at U.~Chicago\\
Carl Rodriguez (2010 -- 2013, co-advised with V.~Kalogera, then F.~Rasio), signatures of deviations from Kerr spacetime; now faculty at Carnegie Mellon University via Pappalardo prize postdoc at MIT\\
Rory Smith (2011 -- 2013), intermediate-mass-ratio waveforms and rapid parameter estimation; now OzGrav postdoc at Monash via postdoc at Caltech\\
Will Vousden (2011 -- 2015), astrophysics with multi-messenger gravitational-wave detections, now at Oxford Asset Management\\
%Kat Grover (2011 -- present, co-advising with A. Vecchio), testing GR with compact-binary coalescences\\
Carl-Johan Haster (2012 -- 2016), topics in gravitational-wave data analysis; now postdoc at MIT via CITA national fellowship, winner of Springer PhD thesis award\\
Simon Stevenson (2013 -- 2017), population synthesis of compact binaries; now OzGrav postdoc at Swinburne University\\
Ben Bradnick (2015 -- 2017, co-advised with W.~Farr), galactic center dynamics; now at Tesella\\
Serena Vinciguerra (2014 -- 2018), advanced techniques for efficient parameter estimation; now postdoc at University of Amsterdam\\
Jim Barrett (2015 -- 2018, co-advised with W.~Farr), emulators and machine learning; now in private-sector big data in Sweden \\
Alejandro Vigna Gomez (2015 -- 2019), mass transfer in stellar binaries; now postdoc at DARK, Copenhagen\\
Coenraad Neijssel (2016 -- 2022), formation of binary black holes across cosmic history\\
Lucy McNeill (2019 -- 2021, co-advised with B.~Mueller), angular momentum transport in evolved stars\\
Reinhold Willcox (2019 -- current), neutron star kicks and dynamics\\
Jeff Riley (2019 -- current), massive stellar binary population synthesis\\
Mike Lau (2019 -- current), hydrodynamics modelling of common envelopes and planetary engulfment\\
Tanner Wilson (2019 -- current, co-advised with A.~Casey), asteroseismology and differential rotation\\
Andreas Konstantinou (2022 -- current), topics in massive binary evolution
\\
{\it Undergraduate and masters students:}\\
Matthew Dodelson (2009) and Daniel Douglas (2010) testing GR with EMRIs\\ 
Frederick Robinson (2009 -- 2010), Fisher matrix analysis\\
Luke Kelley (2010 -- 2011, co-advised with Enrico Ramirez-Ruiz), electromagnetic counterparts to LIGO searches\\
Alex Mellus and Benjamin Hubbert (2012), investigation of numerical waveform accuracy and parameter estimation tests of convergence\\
Jason Tye (2012 -- 2013), animations of gravitational waves, gravitational lensing of gravitational waves\\
Ben Bradnick and Hannah Middleton (2012 -- 2013), progenitors of short gamma ray bursts; shared the Tesella Prize for best computational year 4 project, 2013\\
Zachary Hafen and Jenna Klemkowsky (2013), waveform accuracy requirements on hybrid waveforms and gravitational-wave event rate calculator\\
Kirsty Stroud and Chris Aldridge (2013 -- 2014, co-advised with Will Farr),  exoplanet properties in Kepler data\\
George King and student partner %Hasini Janapriya
(2014 -- 2015, co-advised with Will Farr), searching for exoplanets around red giants\\
Tom Riley, Ben Giblin (2014 -- 2015, co-advised with Will Farr), short gamma ray bursts and time delays in binary mergers; Tom Riley earned prize for top 4th year project\\
Rajath Sathyaprakash and Morgan Brown (2015 -- 2016), massive stellar evolution modeling\\
Andrea Colonna (2015 -- 2016), clustering and inference, jointly with Computer Science\\
Kit Boyer and Fabian Gittins (2016 -- 2017), modelling X-ray binaries\\
Gareth Thomas (2014 -- 2017), searching for intermediate-mass-ratio inspirals\\
Dave Perkins (2016), analysis of population synthesis predictions\\
%George Howitt (2017), visiting PhD student from Australia, luminous red novae\\
Kaila Nathaniel (2017), IREU student, chemically homogeneous evolution\\
Isobel Romero-Shaw (2017), impact of single stellar evolution on binary evolution\\
Luke Nugent (2017), mathematical modeling masters student, efficient population synthesis sampling\\
Alice Perry and Samuel Kingdon (2017-- 2018), Be X-ray binaries as probes of supernovae and binary evolution\\
Nicholas Bennett and Samuel Ratcliff (2017-- 2018, co-advised with Dorottya Sz\'{e}csi), lithium abundances in globular clusters\\ 
Spencer Shortt and Ellen Butler (2018), chemically homogeneous evolution\\
Mike Lau (2018), compact binary observations with LISA\\
Floor Broekgaarden (2019, co-advised with Stephen Justham, Selma de Mink), adaptive importance sampling for population synthesis\\
Michelle Wassink (2019--2020, co-advised with Gijs Nelemans), expansion of stars after the helium main sequence\\
Joanna Shepherd (2020, co-advised with Daniel Price), extracting light curves from hydrodynamical simulations of tidal disruption events\\
Thillai Saravanan (2020--2022), the impact of initial conditions on binary evolution\\
Tushar Nagar (2021, co-advised with Eric Thrane), gravitational-wave source catalog population inference\\
Amir Kashapov(2021, co-advised with Ryo Hirai), late-time expansion of naked helium stars\\
Bayley Tranter (2021, co-advised with Ryo Hirai), few-body scattering experiments\\
Abhi Mangipudi (2021--2022, co-advised with Evgeni Grishin), non-orbit-averaged behaviour in hierarchical triples\\ 
Max Tory (2022, co-advised with Evgeni Grishin), stability of hierarchical triples\\
Andrew Atta (2022, co-advised with Ryo Hirai and Bernhard M\"{u}ller), stripped red supergiants\\
Lewis Picker (2022, co-advised with Ryo Hirai), two-stage common envelopes\\
Aadarsh Madhavan (2022), detectability of luminous red nova progenitors with LSST\\
Alvaro Jose Herrera (2022, co-advised with Ryo Hirai), detached black-hole binaries in Gaia data\\

 


%Supervised undergraduates Matthew Dodelson (NASA summer student, 2009) and Daniel Douglas (2010) on a project regarding the signature of deviations from General Relativity in gravitational waves from extreme-mass-ratio inspirals\\
%Supervised undergraduate Frederick Robinson on a project related to the accuracy of parameter estimation and measurements of an anomalous quadrupole moment during LIGO intermediate-mass-ratio inspirals (2009-2010)\\
%Co-advising graduate students Vivien Raymond (2007 -- present) and Ben Farr (2009 -- present) on applying Markov-Chain Monte-Carlo techniques to LIGO data analysis.\\
%Co-advising graduate student Carl Rodriguez (2010 -- present) on parameter-estimation projects. \\
%Co-advised undergraduate student Luke Kelley (2010 -- 2011) on research on electromagnetic counterparts to LIGO searches \\
%Advising graduate student Will Vousden

\section{\sc Courses taught}
Classical Mechanics and Relativity (first-year, $\gtrsim 200$ students)\\
Waves and Quantum Mechanics (first year, $\sim 200$ students)\\
Inference on Scientific Data (fourth year, $\sim 75$ students)\\
Introduction to General Relativity (third year, $\sim 60$ students)\\
Relativistic Astrophysics and Black Holes (third- and fourth-year, $\sim 50$ students)\\
Introduction to Astrophysics (first-year, $\sim 50$ students)\\ 
Introduction to Particle Physics and Cosmology (first-year, $\sim 50$ students)\\ 
General Physics [back-of-the-envelope physics] (third-year, $\sim 150$ students)\\
Group studies [research project] (third-year, $16$ students)\\
Observational Cosmology (third- and fourth-year, $\sim 30$ students)\\
Gravitational-wave Astrophysics (graduate seminar, $\sim 10$ students)\\
Advanced General Relativity (masters course)\\
Order-of-magnitude Astrophysics (masters course)\\



\section{\sc Grants}
\begin{tabular}{@{}p{0.8in}p{4.7in}}
2024--2030 & Australian Research Council Centre of Excellence OzGrav2 (total of AUD \$35M across OzGrav)\\
2022 & Monash Faculty of Science Strategic Uplift Scheme (CI): {\it Advancing gravitational-wave astrophysics through machine learning} [AUD \$17k]\\
2022--2025 &  Australian Research Council LIEF grant for LSST support (Monash lead CI; total of AUD \$1.685M)\\
2019--2024 & Australian Research Council Future Fellowship (PI): {\it  Shining gravitational waves on binary astrophysics} [AUD \$936k from ARC, AUD \$996k in matching Monash funds]\\
2017--2023 & Australian Research Council Centre of Excellence OzGrav (total of AUD \$35M across OzGrav)\\
2017--2020 & NERC grant (Co-I): {\it Reducing Greenhouse Climate Proxy Uncertainty} [total value of grant \pounds 434k]\\
2016--2019 & STFC Consolidated grant (Co-I): {\it Searching for intermediate mass ratio coalescences in Advanced LIGO/Virgo data} [total value of grant \pounds1,834k]\\
2016--2019 & STFC grant supporting UK Involvement in the Operation of Advanced LIGO (co-I) [\pounds 244k to Birmingham]\\
2016 & RAS Undergraduate Bursary {\it Black holes in globular clusters} (PI) [\pounds 1200]\\
2015 & Kevin Westfold Distinguished Visitor to Monash University, Melbourne, Australia [\$12k support]\\
2015 & RAS Undergraduate Bursary {\it Studying black holes with gravitational waves} (PI) [\pounds 1200]\\
2014 & South African National Institute of Theoretical Physics (NITheP) long-term visitor grant [\$2k]\\
2013--2017 & FP7 Marie Curie Initial Training Network (Co-I): {\it GraWIToN} [total value to Birmingham \pounds 481k]\\
2013--2016 & Leverhulme Trust Research Project Grant (PI): {\it Testing general relativity with ground-based gravitational-wave observations} [\pounds162k]\\
2013--2016 & STFC Consolidated grant (Co-I): {\it Gravitational wave observations of compact binary systems} [total value of grant \pounds1,876k]\\
2013-2016 & ASPERA grant (Co-I): {\it Networking and R\&D for the Einstein Telescope} [salary and travel support, \euro59K]\\
2009--2011 & NSF Astronomy and Astrophysics Postdoctoral Fellowship grant (PI):  {\it Gravitational-wave astronomy: a new window on the universe} [\$249k]\\
2009-2002 & NASA ATP grant (Co-I) {\it Binary White Dwarfs: Gravitational Wave Astrophysics and Data Analysis} [\$399k]\\
\end{tabular}
%Co-I, STFC grant \it{Supporting UK Involvement in the Operation of Advanced LIGO}, 2016--2019\\
%PI, RAS Bursary \it{Studying black holes with gravitational waves}, 2015\\
%Monash University grant for sabbatical visit (Kevin Westfold Distinguished Visitor), 2015\\
%South African National Institute of Theoretical Physics (NITheP) long-term visitor grant, 2014\\
%Co-I, Marie Curie Initial Training Network grant {\it GraWIToN}, 2013--2017\\
%PI, Leverhulme Trust grant, \it{Testing general relativity with ground-based gravitational-wave observations}, 2013--2016\\
%Co-I, STFC consolidated grant {\it Gravitational wave observations of compact binary systems},2013--2016\\
%Co-I, ASPERA grant, {\it Networking and R\&D for the Einstein Telescope}, 2013--2016\\%total amount to Bham -- 59K, including 0.35 months of my salary
%PI on NSF Astronomy and Astrophysics Postdoctoral Fellowship grant ``Gravitational-wave astronomy: a new window on the universe''\\ %[total grant amount over 3 years: \$249,000]\\
%Co-I on NASA ATP grant ``Binary White Dwarfs: Gravitational Wave Astrophysics and Data Analysis''\\




%HiSPARC

\section{\sc Professional Societies}
%American Physical Society\\
%Royal Astronomical Society\\
Astronomical Society of Australia\\
International Society on General Relativity and Gravitation lifetime member\\
International Astrostatistics Association\\
International Astronomical Union


%\section{\sc Personal}
%Born in St. Petersburg, Russia, on June 11, 1979\\
%U.S.~Citizen

\newpage

\section{\sc Publications in Refereed Journals}

\begin{enumerate}

\item I.~Mandel, A.~P.~Igoshev.  2022.  The impact of spin-kick alignment on the inferred velocity distribution of isolated pulsars.  ApJ, accepted. arXiv:2210.12305

\item V.~Kapil, I.~Mandel, E.~Berti, B.~Mueller.  2022.  Calibration of neutron star natal kick velocities to isolated pulsar observations. MNRAS, accepted.  arXiv:2209.09252

\item M.~Tory, E.~Grishin, I.~Mandel. 2022. Empirical Stability Boundary for Hierarchical Triples. PASA, accepted.  arXiv:2208.14005

\item P.~Amaro-Seoane et al. 2022.  Astrophysics with the Laser Interferometer Space Antenna.  Living Reviews in Relativity, accepted.  arXiv:2203.06016

\item R.~Hirai, I.~Mandel. 2022.  A two-stage formalism for common-envelope phases of massive stars.  ApJL 937, L42.  arXiv:2209.05328

\item O.~S.~Salafia, A.~Colombo, F.~Gabrielli, I.~Mandel. 2022. Constraints on the merging binary neutron star mass distribution and equation of state based on the fraction of jets.  A\&A 666, A174. arXiv:2202.01656

\item M.~Lau, R.~Hirai, D.~Price, I.~Mandel.  2022.  Common envelopes in massive stars II: The distinct roles of hydrogen and helium recombination.  MNRAS 516, 4669. arXiv:2206.06411 

\item A.~Mangipudi, E.~Grishin, A.~Trani, I.~Mandel. 2022. Extreme eccentricities of triple systems: Analytic results.  ApJ, accepted.  arXiv:2205.08703

\item F.~Broekgaarden et al.  2022.  Impact of Massive Binary Star and Cosmic Evolution on Gravitational Wave Observations II: Double Compact Object Rates and Properties.  MNRAS, accepted.  arXiv:2112.05763

\item L.~A.~C.~van Son, S.~E.~de Mink, T.~Callister, S.~Justham, M.~Renzo, T.~Wagg, F.~S.~Broekgaarden, F.~Kummer, R.~Pakmor, I.~Mandel.  2022.  The redshift evolution of the binary black hole merger rate: a weighty matter.  ApJ 931, 17.  arXiv:2110.01634 

\item Team COMPAS: J.~Riley et al. 2022.  COMPAS: A rapid binary population synthesis suite.  The Journal of Open Science Software.  https://joss.theoj.org/papers/10.21105/joss.03838

\item I.~Mandel, A.~Farmer.  2022.  Merging stellar-mass binary black holes.  Physics Reports 955, 1.  arXiv:1806.05820

\item A.~Vigna-Gomez, M.~Wassink, J.~Klencki, A.~Istrate, G.~Nelemans, I.~Mandel.  2022. Stellar response after stripping as a model for common-envelope outcomes. MNRAS 511, 2326.  arXiv:2107.14526

\item M.~Y.~M.~Lau, R.~Hirai, M.~Gonz\'alez-Bol\'ivar, D.~J.~Price, O.~De Marco, I.~Mandel. 2022. Common envelopes in massive stars: towards the role of radiation pressure and recombination energy in ejecting red supergiant envelopes. MNRAS 512, 5462.  arXiv:2111.00923

\item I.~Mandel, F.~S.~Broekgaarden. 2022. Rates of compact binary coalescences.  Living Reviews of Relativity 25, 1.  arXiv:2107.14239

\item Team COMPAS: J.~Riley et al. 2022.  Rapid stellar and binary population synthesis with COMPAS.  ApJ Supplements 258, 34.  arXiv:2109.10352

\item I.~Mandel, R.~J.~E.~Smith.  2021.  GW200115: A non-spinning black hole -- neutron star merger.  ApJ Letters 922, L14. arXiv:2109.14759

\item R.~Hirai, I.~Mandel. 2021. Conditions for accretion disc formation and observability of wind-accreting X-ray binaries. PASA 38, e056.  arXiv:2108.03774 

\item S.~Galaudage, C.~Talbot, T.~Nagar, D.~Jain, E.~Thrane, I.~Mandel.  2021. Building better spin models for merging binary black holes: Evidence for non-spinning and rapidly spinning nearly aligned sub-populations. ApJ Letters 921, L15.  arXiv:2109.02424

\item R.~Willcox, I.~Mandel, E.~Thrane, A.~Deller, S.~Stevenson, A.~Vigna-G{\'o}mez.  2021. Constraints on Weak Supernova Kicks from Observed Pulsar Velocities.  ApJ Letters 930, L37.  arXiv:2107.04251 

\item F.~S.~Broekgaarden, E.~Berger, C.~J.~Neijssel, A.~Vigna-Gomez, D.~Chattopadhyay, S.~Stevenson, M.~Chruslinska, S.~Justham, S.~E.~de~Mink, I.~Mandel. 2021. The Impact of Massive Binary Star and Cosmic Evolution on Gravitational Wave Observations I: Black Hole - Neutron Star Mergers.  MNRAS 508, 502. arXiv:2103.02608

\item G.~P.~Lamb, D.~A.~Kann, J.~J.~Fernandez, I.~Mandel, A.~J.~Levan, N.~R.~Tanvir. 2021. GRB jet structure and the jet break. MNRAS 506, 4163.  arXiv:2104.11099

\item R.~Smith, S.~Borhanian, B.~Sathyaprakash, F.~Hernandez Vivanco, S.~Field, P.~Lasky, I.~Mandel, S.~Morisaki, D.~Ottaway, B.~Slagmolen, E.~Thrane, D.~T\"{o}yr\"{a}, S.~Vitale.  2021.  Bayesian inference for gravitational waves from binary neutron star mergers in third-generation observatories.  Phys.~Rev.~Letters 127, 081102. arXiv:2103.12274

\item E.~Grishin, A.~Bobrick, R.~Hirai, I.~Mandel, H.~Perets. 2021.  Supernova explosions in active galactic nuclei discs.  MNRAS 507, 156.  arXiv:2105.09953

\item L.~Lin, D.~Bingham, F.~Broekgaarden, I.~Mandel. 2021. Uncertainty quantification of a computer model for binary black hole formation. Annals of Applied Statistics 15, 1604.  arXiv:2106.01552 

\item J.~Riley, I.~Mandel, P.~Marchant, E.~Butler, K.~Nathaniel, C.~Neijssel, S.~Shortt, A.~Vigna-Gomez.  2021. Chemically Homogeneous Evolution: A rapid population synthesis approach. MNRAS 505, 663.  arXiv:2010.00002

\item J.~C.~A.~Miller-Jones, A.~Bahramian, J.~A.~Orosz, I.~Mandel, L.~Gou, T.~J.~Maccarone, C.~J.~Neijssel et al.  2021.  Cygnus X-1 contains a 21-solar mass black hole -- implications for massive star winds.  Science 371, 1046. arXiv:2102.09091

\item C.~J.~Neijssel, S.~Vinciguerra, A.~Vigna-Gomez, R.~Hirai, J.~C.~A.~Miller-Jones, A.~Bahramian, T.~J.~Maccarone, I.~Mandel.  2021.  Wind mass-loss rates of stripped stars inferred from Cygnus X-1. ApJ 908, 118.  arXiv:2102.09092

\item E.~M.~Rossi, N.~C. Stone, J.~A.~P.~Law-Smith, M.~MacLeod, G.~Lodato, J.~L.~Dai, I.~Mandel.  2021.  The Process of Stellar Tidal Disruption by Supermassive Black Holes. The first pericenter passage.  Space Science Reviews 217, 40.  arXiv:2005.12528

\item D.~Psaltis, C.~Talbot, E.~Payne, I.~Mandel.  2021. Probing the black hole metric. I. Black hole shadows and binary black-hole inspirals.  Phys.~Rev.~D 103, 104036. arXiv:2012.02117

\item C.~Wang, J.~A.~Bendle,  H.~Yang, Y.~Yang, A.~Hardman, A.~Yamoah, A.~Thorpe, I.~Mandel, S.~E.~Greene, J.~Huang.  2021. Global calibration of novel 3-hydroxy fatty acid based temperature and pH proxies.  Geochimica et Cosmochimica Acta 302, 101. 

\item T.~Dunkley Jones, Y.~Eley, W.~Thompson, S.~Greene, I.~Mandel, K.~Edgar, J.~Bendle.  2020.  OPTiMAL: A new machine learning approach for GDGT-based palaeothermometry.  Climate of the Past 16, 2599.  \url{https://cp.copernicus.org/articles/16/2599/2020/cp-16-2599-2020.html}

\item A.~Miglio et al. 2020. Age dissection of the Milky Way discs: red giants in the Kepler field.  A\&A, accepted.  arXiv:2004.14806

\item I.~Mandel, B.~M\"{u}ller, J.~Riley, S.~E.~de Mink, A.~Vigna-Gomez, D.~Chattopadhyay. 2021.  Binary population synthesis with probabilistic remnant mass and kick prescriptions. MNRAS 500, 1380.   arXiv:2007.03890

\item L.~du Buisson, P.~Marchant, P.~Podsiadlowski, C.~Kobayashi, F.~B.~Abdalla, P.~Taylor, I.~Mandel, S.~E.~de Mink, T.~J.~Moriya, N.~Langer.  2020. Cosmic Rates of Black Hole Mergers and Pair-Instability Supernovae from Chemically Homogeneous Binary Evolution. MNRAS, 499, 5941.  arXiv:2002.11630

\item I.~Mandel and B.~M\"{u}ller. 2020. Simple recipes for compact remnant masses and natal kicks.  MNRAS 499, 3214.  arXiv:2006.08360

\item K.~Ackley et al. 2020. Neutron Star Extreme Matter Observatory: A kilohertz-band gravitational-wave detector in the global network. PASA, accepted.  arXiv:2007.03128 

\item R.~Hirai, T.~Sato, P.~Podsiadlowski, A.~Vigna-Gomez, I.~Mandel. 2020. Formation pathway for lonely stripped-envelope supernova progenitors: implications for Cassiopeia A.  MNRAS 499, 1154. arXiv:2008.05076

\item A.~Murguia-Berthier, A.~Batta, A.~Janiuk, E.~Ramirez-Ruiz, I.~Mandel, S.~C.~Noble, R.~W.~Everson.  2020.  On the maximum stellar rotation to form a black hole without an accompanying luminous transient.  ApJ Letters 901, L24.  arXiv:2005.10212

\item M.~Nicholl et al. 2020. An outflow powers the optical rise of the nearby, fast-evolving tidal disruption event AT2019qiz. MNRAS 499, 482.  arXiv:2006.02454

\item K.~Ackley et al. 2020. Observational constraints on the optical and near-infrared emission from the neutron star-black hole binary merger S190814bv.  Astronomy \& Astrophysics 643, A113.
arXiv:2002.01950

\item A.~Vigna-Gomez, M.~MacLeod, C.~J.~Neijssel, F.~S.~Broekgaarden, S.~Justham, G.~Howitt, S.~E.~de Mink and I.~Mandel.  2020. Common-Envelope Episodes that lead to Double Neutron Star formation.  PASA 37, e038.  arXiv:2001.09829 

\item S.~Vinciguerra, C.~J.~Neijssel, A.~Vigna-G\'{o}mez, I.~Mandel, P.~Podsiadlowski, T.~J.~Maccarone, M.~Nicholl, S.~Kingdon, A.~Perry, F.~Salemi.  2020. Be X-ray binaries in the SMC as (I) indicators of mass transfer efficiency.  MNRAS 498, 4705.  arXiv:2003.00195

\item V.~Korol, I.~Mandel, M.~C.~Miller, R.~P.~Church, M.~B.~Davies. 2020. Merger rates in primordial black hole clusters without initial binaries. MNRAS 496, 994.  arXiv:1911.03483

\item S.~De, M.~MacLeod, R.~W.~Everson, A.~Antoni, I.~Mandel, E.~Ramirez-Ruiz.  2020.  Common Envelope Wind Tunnel: The Effects of Binary Mass Ratio and Implications for the Accretion-Driven Growth of LIGO Binary Black Holes.  ApJ 897, 130.  arXiv:1910.13333

\item S.~L.~Schroeder, M.~MacLeod, A.~Loeb, A.~Vigna-Gomez, I.~Mandel.  2020. Explosions Driven by the Coalescence of a Compact Object with the Core of a Massive-Star Companion Inside a Common Envelope: Circumstellar Properties, Light Curves, and Population Statistics.  ApJ 892, 13.  arXiv:1906.04189

\item I.~Mandel, T.~Fragos. 2020. An alternative interpretation of GW190412 as a binary black hole merger with a rapidly spinning secondary.  ApJ Letters 895, L28. arXiv:2004.09288

\item S.~S.~Bavera, T.~Fragos, Y.~Qin, E.~Zapartas, C.~J.~Neijssel, I.~Mandel, A.~Batta, S.~M.~Gaebel, C.~Kimball, S.~Stevenson.  2020.  The origin of spin in binary black holes: Predicting the distributions of the main observables of Advanced LIGO.  Astronomy \& Astrophysics 635, A97.  arXiv:1906.12257

\item M.~Y.~M.~Lau, I.~Mandel, A.~Vigna-G\'{o}mez, C.~J.~Neijssel, S.~Stevenson, A.~Sesana.  2020.  Detecting Double Neutron Stars with LISA.  MNRAS 492, 3061.  arXiv:1910.12422

\item G.~Howitt, S.~Stevenson, A.~Vigna-Gomez, S.~Justham, N.~Ivanova, T.~E.~Woods, C.~J.~Neijssel, I.~Mandel.  2020.  Luminous Red Novae: population models and future prospects.  MNRAS 492, 3229.  arXiv:1912.07771 

\item P.~Marchant, K.~Breivik, C.~P.~L.~Berry, I.~Mandel, S.~L.~Larson. 2020.  Eclipses of continuous gravitational waves as a probe of stellar structure.  Physical Review D 101, 024039.
arXiv: 1912.04268

\item L.~Wyrzykowski, I.~Mandel.  2020.  Constraining the masses of microlensing black holes and the mass gap with Gaia DR2.  Astronomy \& Astrophysics 636, A20.  arXiv:1904.07789

\item O.~R.~McBrien et al. 2019. SN2018kzr: a rapidly declining transient from the destruction of a white dwarf. ApJ Letters 885, L23.  arXiv:1909.04545

\item G.~Leloudas et al.  2019. The spectral evolution of AT 2018dyb and the presence of metal lines in tidal disruption events.  ApJ 887, 218.  arXiv:1903.03120

\item C.~J.~Neijssel, A.~Vigna-Gomez, S.~Stevenson, J.~W.~Barrett, S.~M.~Gaebel, F.~Broekgaarden, S.~E.~de Mink, D.~Szecsi, S.~Vinciguerra, I.~Mandel.  2019. The effect of the metallicity-specific star formation history on double compact object mergers.  MNRAS 490, 3740.  arXiv:1906.08136

\item F.~S.~Broekgaarden, S.~Justham, S.~E.~de Mink, J.~Gair, I.~Mandel, S.~Stevenson, J.~W.~Barrett, A.~Vigna-Gomez, C.~J.~Neijssel.  2019. STROOPWAFEL: Simulating rare outcomes from astrophysical populations, with application to gravitational-wave sources.  MNRAS 490, 5228.  arXiv:1905.00910

\item R.~X.~Adhikari, P.~Ajith, Y.~Chen, J.~A.~Clark, V.~Dergachev, N.~V.~Fotopoulos, S.~E.~Gossan, I.~Mandel, M.~Okounkova, V.~Raymond, J.~S.~Read.  2019. Astrophysical science metrics for next-generation gravitational-wave detectors.  CQG 36, 245010.  arXiv:1905.02842

\item J.~J.~Andrews, I.~Mandel. 2019. Double Neutron Star Populations and Formation Channels.   2019.  ApJ Letters 880, L8.  arXiv:1904.12745

\item J.~Powell, S.~Stevenson, I.~Mandel, P.~Tino.  2019.  Unmodelled Clustering Methods for Gravitational Wave Populations of Compact Binary Mergers.  MNRAS 488, 3810. arXiv:1905.04825

\item S.~Stevenson, M.~Sampson, J.~Powell, A.~Vigna-G\'{o}mez, C.~J.~Neijssel, D.~Sz\'{e}csi, I.~Mandel.  2019. The impact of pair-instability mass loss on the binary black hole mass distribution.   ApJ 882, 121.  arXiv:1904.02821 

\item A.~Vigna-Gomez, S.~Justham, I.~Mandel, S.~E.~de Mink, P.~Podsiadlowski.  2019.  Massive Stellar Mergers as Precursors of Hydrogen-rich Pulsational Pair Instability Supernovae.  ApJ Letters 876, L29. arXiv:1903.02135

\item I.~Mandel, W.~M.~Farr, J.~R.~Gair.  2019. Extracting distribution parameters from multiple uncertain observations with selection biases.  MNRAS 486, 1086.  arXiv:1809.02063

\item G.~P.~Lamb et al. 2019. The optical afterglow of GW170817 at one year post-merger.  ApJ Letters 870, L15. arXiv:1811.11491

\item S.~Vinciguerra, M.~Branchesi, R.~Ciolfi, I.~Mandel, C.~Neijssel, G.~Stratta.  2019. saprEMo: a simplified algorithm for predicting detections of electromagnetic transients in surveys.  MNRAS 484, 332.  arXiv:1809.08641

\item G.~P.~Lamb, I.~Mandel, L.~Resmi.  2018.  Late-time Evolution of Afterglows from Off-Axis Neutron-Star Mergers.  MNRAS 481, 2581.  arXiv:1806.03843

\item A.~Vigna-G\'{o}mez, C.-J.~Neijssel, S.~Stevenson, J.~W.~Barrett, K.~Belczynski, S.~Justham, S.E.~de Mink, B.~M{\"u}ller, P.~Podsiadlowski, M.~Renzo, D.~Sz{\'e}csi, I.~Mandel.  2018. On the formation history of Galactic double neutron stars.  MNRAS 481, 4009. arXiv:1805.07974

\item W.~M.~Farr, I.~Mandel.  2018.  Comment on ``An excess of massive stars in the local 30 Doradus starburst''.  Science 361, 6400.  arXiv:1807.09772

\item J.~D.~Lyman, G.~P.~Lamb, A.~J.~Levan, I.~Mandel, N.~R.~Tanvir et al.  2018. The optical afterglow of the short gamma-ray burst associated with GW170817.  Nature Astronomy. arXiv:1801.02669

\item J.~W. Barrett, S.~M.~Gaebel, C.~J.~Neijssel, A.~Vigna-G\'{o}mez, S.~Stevenson, C.~P.~L.~Berry, W.~M.~Farr, I.~Mandel.  2018.  Accuracy of inference on the physics of binary evolution from gravitational-wave observations.   MNRAS 477, 4685.  arXiv:1711.06287

\item M.~Cantiello et al. 2018.  A precise distance to the host galaxy of the binary neutron star merger GW170817 using surface brightness fluctuations.  ApJL 854, L31.  arXiv:1801.06080

\item I.~Mandel.  2018. The orbit of GW170817 was inclined by less than 28 degrees to the line of sight.  ApJL 853, L1. arXiv:1712.03958

\item I.~Mandel, A.~Sesana, A.~Vecchio. 2018.  The astrophysical science case for a decihertz gravitational-wave detector.  Classical and Quantum Gravity 35, 054004. arXiv:1710.11187

\item N.~R.~Tanvir, A.~J.~Levan, C.~Gonzalez-Fernandez, O.~Korobkin, I.~Mandel et al. 2017.  The Emergence of a Lanthanide-rich Kilonova Following the Merger of Two Neutron Stars.  ApJL 848, L27.  arXiv:1710.05455 

\item A.~J.~Levan, J.~D.~Lyman, N.~R.~Tanvir, J.~Hjorth, I.~Mandel et al. 2017. The environment of the binary neutron star merger GW170817.  ApJL 848, L28.  arXiv:1710.05444

\item J.~Hjorth et al. 2017. The Distance to NGC 4993: The Host Galaxy of the Gravitational-wave Event GW170817.  ApJL 848, L31.  arXiv:1710.05866

\item B.~P.~Abbott et al. 2017.   A gravitational-wave standard siren measurement of the Hubble constant.  Nature 551, 85. arXiv:1710.05835

\item B.~P.~Abbott et al. 2017. Multi-messenger observations of a binary neutron star merger. ApJL 848, L12. arXiv:1710.05833

\item I.~Mandel, A.~Farmer.  2017. Gravitational waves: Stellar palaeontology.  Nature 547, 284

\item W.~M.~Farr, S.~Stevenson, M.~C.~Miller, I.~Mandel, B.~Farr, A.~Vecchio.  Distinguishing Spin-Aligned and Isotropic Black Hole Populations With Gravitational Waves. 2017. Nature, 548, 426. arXiv:1706.01385

\item P.~Marchant, N.~Langer, P.~Podsiadlowski, T.~Tauris, S.~de Mink, I.~Mandel, T.~Moriya.  2017.  Ultra-luminous X-ray sources and neutron-star-black-hole mergers from very massive close binaries at low metallicity.  A\&A 604, A55.  arXiv:1705.04734

\item S.~Stevenson, A.~Vigna-G\'omez, I.~Mandel, J.~W.~Barrett, C.~J.~Neijssel, D.~Perkins, S.~E.~de Mink.  2017.  Forming GW151226 and LVT151012 through isolated binary evolution.  Nature Communications, 8,  14906.  arXiv:1704.01352

\item S.~Stevenson, C.~P.~L.~Berry, I.~Mandel.  2017.  Hierarchical analysis of gravitational-wave measurements of binary black hole spin-orbit misalignments.  MNRAS 471, 2801. arXiv:1703.06873

\item B.~Bradnick, I.~Mandel, Y.~Levin. 2017.  Stellar binaries in galactic nuclei: tidally stimulated mergers followed by tidal disruptions.  MNRAS 469, 2042.  arXiv:1703.05796

\item S.~Vinciguerra, J.~Veitch, I.~Mandel.  2017.  Accelerating gravitational wave parameter estimation with multi-band template interpolation.  CQG, 34, 115006.  arXiv:1703.02062

\item I.~Mandel, W.~M.~Farr, A.~Colonna, S.~Stevenson, P.~Ti\v{n}o, J.~Veitch.  2017.  Model-independent inference on compact-binary observations.  MNRAS 465, 3254.  arXiv:1608.08223

\item C.-J.~Haster, F.~Antonini, V.~Kalogera, I.~Mandel. 2016. N-body dynamics of intermediate mass-ratio inspirals in globular clusters. ApJ 832, 192.  arXiv:1606.07097

\item L.~P.~Singer et al. 2016.  Going the Distance: Mapping Host Galaxies of LIGO Sources in Three Dimensions Using Local Cosmography and Targeted Follow-up.  ApJL, 829, L15. arXiv:1603.07333.   Supplementary material in ApJS, 226, 10, arXiv:1605.04242

\item S.~J.~Smartt et al.  2016.  A search for an optical counterpart to the gravitational wave event GW151226.  ApJL, 827, L40. arXiv:1606.04795

\item T.~Callister, L.~Sammut, E.~Thrane, S.~Qiu, I.~Mandel.  2016. The limits of astrophysics with gravitational wave backgrounds. Phys.~Rev.~X 6, 031018.  arXiv:1604.02513

\item S.~E.~de Mink, I.~Mandel. 2016. The Chemically Homogeneous Evolutionary Channel for Binary Black Hole Mergers: Rates and Properties of Gravitational-Wave Events Detectable by Advanced LIGO.  MNRAS 460, 3545.  arXiv:1603.02291

\item I.~Mandel, S.~E.~de Mink.  2016.  Merging binary black holes formed through chemically homogeneous evolution in short-period stellar binaries. MNRAS 458, 2634. arXiv:1601.00007

\item P.~A.~Rosado, P.~D.~Lasky, E.~Thrane, X.~Zhu, I.~Mandel, A.~Sesana.  2016. The most distant observable massive objects.  Phys.~Rev.~Lett. 116, 101102.  arXiv:1512.04950

\item B.~Farr et al. 2016. Parameter estimation on gravitational waves from neutron-star binaries with spinning components. ApJ 825, 116. arXiv:1508.05336

\item S.~A.~L.~Otaibi, P.~Tino, J.~Cuevas-Tello, I.~Mandel, S.~Raychaudhury.  2016.  Kernel regression estimates of time delays between gravitationally lensed fluxes.  MNRAS 459, 573.  arXiv:1508.03439

\item C.~Haster, Z.~Wang, C.~Berry, S.~Stevenson, J.~Veitch, I.~Mandel.  2016. Inference on gravitational waves from coalescences of stellar-mass compact objects and intermediate-mass black holes.   MNRAS 457, 4499.  arXiv:1511.01431 

\item I.~Mandel. 2016. Estimates of black-hole natal kick velocities from observations of low-mass X-ray binaries. MNRAS, 456, 578. arXiv:1510.03871

\item  I.~Mandel and Y.~Levin.  2015.  Double tidal disruptions in galactic nuclei.  ApJ Letters, 805, L4.  arXiv:1504.02787

\item  J.~Veitch, M.~P\"urrer, I.~Mandel. 2015. Measuring intermediate mass black hole binaries with advanced gravitational wave detectors. PRL, 115, 141101. arXiv:1503.05953 

\item  I.~Mandel, C.-J.~Haster, M.~Dominik, K.~Belczynski.  2015.  Distinguishing types of compact-object binaries using the gravitational-wave signatures of their mergers.  MNRAS Letters, 450, L85. arXiv:1503.03172

\item C.~W.~F.~Everitt et al.  2015.  The Gravity Probe B test of general relativity.  Classical and Quantum Gravity, 32 224001.

\item A.~S.~Silbergleit, J.~W.~Conklin, M.~I.~Heifetz, T.~Holmes, J.~Li, I.~Mandel, et al. 2015.
Gravity Probe B data analysis: II. Science data and their handling prior to the final analysis.  Classical and Quantum Gravity, 32 224019.

\item  C.-J.~Haster, I.~Mandel,  W.~M.~Farr.  2015.  Efficient method for measuring the parameters encoded in a gravitational-wave signal.  Classical and Quantum Gravity, 32, 235017. arXiv:1502.05407

\item W.~Vousden, W.~M.~Farr, I.~Mandel.  2016. Dynamic temperature selection for parallel-tempering in Markov chain Monte Carlo simulations.  MNRAS 455, 1919. arXiv:1501.05823 

\item  C.~Berry, I.~Mandel, et al. 2015. Parameter estimation for binary neutron-star coalescences with realistic noise during the Advanced LIGO era.  ApJ, 804, 114.  arXiv:1411.6934

\item  J.~Veitch et al. 2015. Robust parameter estimation for compact binaries with ground-based gravitational-wave observations using LALInference. Phys.~Rev.~D 91, 042003. arXiv:1409.7215

\item  W.~Del Pozzo, K.~Grover, I.~Mandel, A.~Vecchio. 2014.  Testing general relativity with compact coalescing binaries: comparing exact and predictive methods to compute the Bayes factor. Class.~Quantum Grav.~ 31, 205006. arXiv:1408.2356

\item  M.~Dominik, E.~Berti, R.~O'Shaughnessy, I.~Mandel, K.~Belczynski, C.~Fryer, D.~Holz, T.~Bulik,  F.~Pannarale. 2015. Double Compact Objects III: Gravitational Wave Detection Rates.  ApJ, 806, 263.  arXiv:1405.7016

\item  L.~Singer et al. 2014. The First Two Years of Electromagnetic Follow-Up with Advanced LIGO and Virgo.  ApJ, 795, 105. arXiv:1404.5623

\item  I.~Mandel, C.~P.~L.~Berry, F.~Ohme, S.~Fairhurst, W.~Farr.  2014. Parameter estimation on compact binary coalescences with abruptly terminating gravitational waveforms.  Class.~Quantum Grav.~31 155005. arXiv:1404.2382

\item  K.~Belczynski, A.~Buonanno, M.~Cantiello, C.~Fryer, D.~Holz, I.~Mandel, M.~C.~Miller, M.~Walczak.  2014. The Formation and Gravitational-Wave Detection of Massive Stellar Black-Hole Binaries.  ApJ 789, 120. arXiv:1403.0677

\item  T.~Sidery et al. 2014. Reconstructing the sky location of gravitational-wave detected compact binary systems: methodology for testing and comparison.  Phys.~Rev.~D 89, 084060.  arXiv:1312.6013

\item  C.~Hanna, I.~Mandel, W.~Vousden.  2014. Utility of galaxy catalogs for following up gravitational waves from binary neutron star mergers with wide-field telescopes. ApJ, 784, 8.  arXiv:1312.2077

\item  K.~Grover, S.~Fairhurst, B.~F.~Farr, I.~Mandel, C.~Rodriguez, T.~Sidery, A.~Vecchio.  2014.  Comparison of Gravitational Wave Detector Network Sky Localization Approximations.  Phys.~Rev.~D 89, 042004.  arXiv:1310.7454

\item  M.~Dominik, K.~Belczynski, C.~Fryer, D.~Holz, E.~Berti, T.~Bulik, I.~Mandel, R.~O'Shaughnessy. 2013.  Double Compact Objects II: Cosmological Merger Rates.  ApJ 779, 72. arXiv:1308.1546

\item  C.~L.~Rodriguez, B.~Farr, W.~M.~Farr, I.~Mandel.  2013.  Inadequacies of the Fisher Information Matrix in gravitational-wave parameter estimation.  Phys.~Rev.~D 88, 084013.  arXiv:1308.1397 

\item  C.~M.~F.~Mingarelli, T.~Sidery, I.~Mandel, A.~Vecchio.  2013.  Characterising gravitational wave stochastic background anisotropy with Pulsar Timing Arrays.  Phys.~Rev.~D 88, 062005. arXiv:1306.5394

\item  R.~J.~E.~Smith, C.~Hanna, I.~Mandel, A.~Vecchio.  2014.  Rapidly evaluating the compact binary likelihood function via interpolation.  Phys.~Rev.~D 90, 044074. arXiv:1305.3798

\item  N.~Andersson et al.  2013.  The Transient Gravitational-Wave Sky.  arXiv:1305.0816

\item  R.~J.~E.~Smith, I.~Mandel, A.~Vecchio. 2013.  Studies of waveform requirements for intermediate mass-ratio coalescence searches with advanced detectors. Phys.~Rev.~D 88, 044010.  arXiv:1302.6049

\item  W.~M.~Farr, J.~R.~Gair, I.~Mandel, C.~Cutler.  2015.  Counting And Confusion: Bayesian Rate Estimation With Multiple Populations.  Phys.~Rev.~D 91, 023005. arXiv:1302.5341

\item  R.~J.~E.~Smith, K.~Cannon, C.~Hanna, D.~Keppel, I.~Mandel. 2013. Towards Rapid Parameter Estimation on Gravitational Waves from Compact Binaries using Interpolated Waveforms.  Phys.~Rev.~D 87, 122002.  arXiv:1211.1254

\item  L.~Z.~Kelley, I.~Mandel, E.~Ramirez-Ruiz. 2013.  Electromagnetic transients as triggers in searches for gravitational waves from compact binary mergers. Phys.~Rev.~D 87, 123004.  arXiv:1209.3027

\item  K.~Belczynski, T.~Bulik, I.~Mandel, B.~S.~Sathyaprakash, A.~Zdziarski, J.~Mikolajewska. 2013. Cyg X-3: a Galactic double black hole or black hole-neutron star progenitor. ApJ 764 96. arXiv:1209.2658

\item  J.~Chennamangalam, D.~R.~Lorimer, I.~Mandel, M.~Bagchi.  2013. Constraining the luminosity function parameters and population size of radio pulsars in globular clusters.  MNRAS 431, 874.  arXiv:1207.5732, arXiv:1210.5472

\item  M.~Dominik, K.~Belczynski, C.~Fryer, D.~Holz, E.~Berti, T.~Bulik, I.~Mandel, R.~O'Shaughnessy.  2012.  Double Compact Objects I: The Significance Of The Common Envelope On Merger Rates.  ApJ 759, 52.  arXiv:1202.4901

\item  J.~Veitch, I.~Mandel, B.~Aylott, B.~Farr, V.~Raymond, C.~Rodriguez, M.~van der Sluys, V.~Kalogera, A.~Vecchio.  2012.  Estimating parameters of coalescing compact binaries with proposed advanced detector networks. Phys.~Rev.~D 85, 104045. arXiv:1201.1195

\item  C.~L.~Rodriguez, I.~Mandel, J.~R.~Gair.  2012. Verifying the no-hair property of massive compact objects with intermediate-mass-ratio-inspirals in advanced gravitational-wave detectors. Phys.~Rev.~D 85, 062002. arXiv:1112.1404

\item  S.~Vitale, W.~Del Pozzo, T.~G.~F.~Li, C.~Van Den Broeck, I.~Mandel, B.~Aylott, J.~Veitch.  2011.  Effect of calibration errors on Bayesian parameter estimation for gravitational wave signals from inspiral binary systems in the Advanced Detectors era.  Phys.~Rev.~D 85, 064034. arXiv:1111.3044
 
\item  S.~R.~Taylor, J.~R.~Gair, I.~Mandel. 2012. Hubble without the Hubble: cosmology using advanced gravitational-wave detectors alone. Phys.~Rev.~D 85:023535. arXiv:1108.5161 

\item  W.~Farr, I.~Mandel, D.~Stevens. 2015.  An Efficient Interpolation Technique for Jump Proposals in Reversible-Jump Markov Chain Monte Carlo Calculations.  Royal Society Open Science, 2, 150030. arXiv:1104.0984

\item  W.~Farr, N.~Sravan, A.~Cantrell, L.~Kreidberg, C.~D.~Bailyn, I.~Mandel, V.~Kalogera. 2011. The Mass Distribution of Stellar-Mass Black Holes. ApJ 741, 103. arXiv:1011.1459

\item  L.~Z.~Kelley, E.~Ramirez-Ruiz, M.~Zemp, J.~Diemand, I.~Mandel. 2010. The Distribution of Coalescing Compact Binaries in the Local Universe: Prospects for Gravitational-Wave Observations. ApJL 725, L91.  arXiv:1011.1256

\item  M.~Punturo et al. 2010. Third generation of gravitational wave observatories and their science reach.  Class.~Quantum Grav.~27, 084007.

\item  I.~Mandel. 2010.  Parameter estimation on gravitational waves from multiple coalescing binaries.  Phys.~Rev.~D 81, 084029. arXiv: 0912.5531

\item  V.~Raymond, M.~V.~van der Sluys, I.~Mandel, V.~Kalogera, C.~Roever, N.~Christensen. 2010. The effects of LIGO detector noise on a 15-dimensional Markov-chain Monte-Carlo analysis of gravitational-wave signals.  Class.~Quantum Grav.~27, 114009. arXiv:0912.3746

\item  I.~Mandel and R.~O'Shaughnessy.  2010.  Compact Binary Coalescences in the Band of Ground-based Gravitational-Wave Detectors.  Class.~Quantum Grav.~27, 114007. arXiv:0912.1074  

\item  S.~Babak et al.~(Mock LISA Data Challenge Team). 2010.  The Mock LISA Data Challenges: from Challenge 3 to Challenge 4 Class.~Quantum Grav.~27, 084009.  arXiv:0912.0548  

\item  J.~Gair, I.~Mandel, M.~C.~Miller, M.~Volonteri.  2011.  Exploring intermediate and massive black-hole binaries with the Einstein Telescope. General Relativity and Gravitation, 43, 485-518.  arXiv:0907.5450

\item  J.~Gair, I.~Mandel, A.~Sesana, A.~Vecchio.  2009. Probing seed black holes using the next generation of gravitational-wave detectors.  Class.~Quant.~Grav.~ 26, 204009.  arXiv:0907.3292

\item  L.~Cadonati et al.~(NINJA collaboration). 2009. Status of NINJA: the numerical INJection Analysis project.  Class.~Quant.~Grav.~26, 114008. arXiv:0905.4227

\item  M.~van der Sluys, I.~Mandel, V.~Raymond, V.~Kalogera, C.~Roever, N.~Christensen. 2009. Parameter estimation for signals from compact binary inspirals injected into LIGO data.  Class.~Quant.~Grav.~ 26, 204010. arXiv:0905.1323

\item  A.~Sesana, J.~Gair, I.~Mandel, A.~Vecchio. 2009. Observing gravitational waves from the first generation of black holes.  ApJL 698, L129. arXiv:0903.4177 

\item  B.~Aylott et al.~(NINJA collaboration).  2009. Testing gravitational-wave searches with numerical relativity waveforms: Results from the first Numerical INJection Analysis (NINJA) project.  Class.~Quant.~Grav.~26, 165008.  arXiv:0901.4399

\item  V.~Raymond, M.~van der Sluys, I.~Mandel, V.~Kalogera, C. Roever, N. Christensen.  2009. Degeneracies in Sky Localisation Determination from a Spinning Coalescing Binary through Gravitational Wave Observations: a Markov-Chain Monte-Carlo Analysis for two Detectors. Class.~Quant.~Grav.~26, 114007.  arXiv:0812.4302

\item  K.~G.~Arun et al.~(LISA PE taskforce). 2009.  Massive Black Hole Binary Inspirals: Results from the LISA Parameter Estimation Taskforce.  Class.~Quant.~Grav.~26, 094027.  arXiv: 0811.1011

\item  I.~Mandel, J.~R.~Gair.  2009.  Can we Detect Intermediate Mass Ratio Inspirals?  Class.~Quant.~Grav.~26, 094036.  arXiv:0811.0138

\item  M.~van der Sluys, C.~Roever, A.~Stroeer, V.~Raymond, I.~Mandel, N.~Christensen, 
V.~Kalogera, R.~Meyer, A.~Vecchio. 2008.  Gravitational-Wave Astronomy with Inspiral Signals of Spinning Compact-Object Binaries.  ApJL 688, L61. arXiv:0710.1897

\item  S.~Babak et al.~(Mock LISA Data Challenge Team). 2008. The Mock LISA Data Challenges: from Challenge 1B to Challenge 3.  Class.~Quant.~Grav.~25, 184026.
arXiv:0806.2110 

\item  M.~van der Sluys, V.~Raymond, I.~Mandel, C.~Roever, N.~Christensen, 
V.~Kalogera, R.~Meyer, A.~Vecchio.  2008. Parameter Estimation of 
Spinning Binary Inspirals Using Markov-Chain Monte Carlo.  Class.~Quant.~Grav.~25, 184011.
arXiv:0805.1689

\item  J.~R.~Gair, I.~Mandel, L.~Wen. 2008. Improved Time-Frequency Analysis of 
Extreme-Mass-Ratio Inspiral Signals in Mock LISA Data. Class.~Quant.~Grav.~25, 184031.
arXiv:0804.1084

\item  S.~Babak et al.~(Mock LISA Data Challenge Team). 2008. Report on the 
Second Mock LISA Data Challenge.  Class.~Quant.~Grav.~25, 114037.
arXiv:0711.2667

\item  J.~R.~Gair, C.~Li, I.~Mandel. 2008. Observable Properties of Orbits in 
Exact Bumpy Spacetimes.  Phys. Rev. D 77 024035.  arXiv:0708.0628

\item  I.~Mandel, D.~A.~Brown, J.~R.~Gair, M.~C.~Miller. 2008. Rates and 
Characteristics of Intermediate-Mass-Ratio Inspirals Detectable by 
Advanced LIGO.  ApJ 681 1431-1447. arXiv:0705.0285

\item  D.~A.~Brown, J.~Crowder, C.~Cutler, I.~Mandel, M.~Vallisneri. 2007.  A 
Three-Stage Search for Supermassive Black Hole Binaries in LISA data. 
Class.~Quant.~Grav.~24, S595-S605. arXiv:0704.2447

\item  P.~Amaro-Seoane, J.~R.~Gair, M.~Freitag, M.~C.~Miller, I.~Mandel,
C.~Cutler, S.~Babak. 2007.  Intermediate and Extreme Mass-Ratio  
Inspirals -- Astrophysics, Science Applications and Detection using
LISA.  Class.~Quant.~Grav.~24 R113-R170. arXiv:astro-ph/0703495

\item  K.~Arnaud et al.~(Mock LISA Data Challenge Team).  2007. Report on the 
First Round of the Mock LISA Data Challenges.  Class.~Quant.~Grav.~24, 
S529-S539. arXiv:gr-qc/0701139

\item  D.~A.~Brown, J.~Brink, H.~Fang, J.~R.~Gair, C.~Li, G.~Lovelace, 
I.~Mandel, K.~S.~Thorne. 2007.  Gravitational Waves from 
Intermediate-Mass-Ratio Inspirals for Ground-based Detectors. Phys.~Rev.~Lett.~99, 201102. arXiv:gr-qc/0612060

\item  I.~Mandel. 2005. The Geometry of a Naked Singularity Created by Standing 
Waves Near a Schwarzschild Horizon, and Its Application to the Binary 
Black Hole Problem. Phys.~Rev.~D 72 084025.  arXiv:gr-qc/0505149

\item  A.~Silbergleit, I.~Mandel, I.~Nemenman.  2003.  Potential and Field 
Singularity at a Surface Point Charge.  J.~Math.~Phys.~44 (10) 
4460-4466.  arXiv:math-ph/0306039

%\end{enumerate}
\vspace{0.2in}
{\bf \large Publications with the LIGO Scientific Collaboration (LSC) [member 2005--2016]:}

%\begin{enumerate}
%\setcounter{enumi}{63}

\item Abbott, B. et al. 2016.  GW151226: Observation of Gravitational Waves from a 22-Solar-Mass Binary Black Hole Coalescence.  Phys.~Rev.~Letters 116, 241103.  	arXiv:1606.04855

\item Abbott, B. et al. 2016.  Binary Black Hole Mergers in the first Advanced LIGO Observing Run.  Phys.~Rev.~X 6, 041015.  	arXiv:1606.04856

\item Abbott, B. et al. 2016.  Observation of Gravitational Waves from a Binary Black Hole Merger.  Phys.~Rev.~Letters 116, 061102.  arXiv:1602.03837

\item Abbott, B. et al. 2016.  Astrophysical Implications of the Binary Black-hole Merger GW150914.  ApJL 818, L22. arXiv:1602.03846

\item Abbott, B. et al. 2016. Tests of general relativity with GW150914. Phys.~Rev.~Letters 116, 221101. arXiv:1602.03841

\item Abbott, B. et al. 2016.  {\it (as internal reviewer)}   GW150914: Implications for the stochastic gravitational wave background from binary black holes.  Phys.~Rev.~Letters 116, 131102.  arXiv:1602.03847

\item Abbott, B. et al. 2016. GW150914: The Advanced LIGO Detectors in the Era of First Discoveries.   Phys.~Rev.~Letters 116, 131103.  arXiv:1602.03838

\item Abbott, B. et al. 2016. Properties of the binary black hole merger GW150914.   Phys.~Rev.~Letters, 116, 241102.  arXiv:1602.03840

\item Abbott, B. et al. 2016. GW150914: First results from the search for binary black hole coalescence with Advanced LIGO.  Phys.~Rev.~D 93, 122003.  arXiv:1602.03839

\item Abbott, B. et al. 2017. Calibration of the Advanced LIGO detectors for the discovery of the binary black-hole merger GW150914.   Phys.~Rev.~D 95, 062003.  arXiv:1602.03845

\item Abbott, B. et al. 2016. Localization and broadband follow-up of the gravitational-wave transient GW150914.   ApJL 825, L13. arXiv:1602.08492

\item Abbott, B. et al. 2016. Observing gravitational-wave transient GW150914 with minimal assumptions.  Phys.~Rev.~D 93, 122004. arXiv:1602.03843

\item Abbott, B. et al. 2016. The Rate of Binary Black Hole Mergers Inferred from Advanced LIGO Observations Surrounding GW150914. ApJL 833, L1.  arXiv:1602.03842  [Supplement in ApJS 227, 2, 14. arXiv:1606.03939]

\item Abbott, B. et al. 2016.  High-energy Neutrino follow-up search of Gravitational Wave Event GW150914 with ANTARES and IceCube.   Phys.~Rev.~D 93, 122010. arXiv:1602.05411

\item Abbott, B. et al. 2016. Characterization of transient noise in Advanced LIGO relevant to gravitational wave signal GW150914.   CQG 33, 134001. arXiv:1602.03844

\item Aasi, J. et al. 2015. A directed search for gravitational waves from Scorpius X-1 with initial LIGO.  Phys. Rev. D 91, 062008

\item Aasi, J. et al. 2015. Narrow-band search of continuous gravitational-wave signals from Crab and Vela pulsars in Virgo VSR4 data.  Phys. Rev. D 91, 022004

\item Aasi, J. et al. 2015. Characterization of the LIGO detectors during their sixth science run. Class. Quantum Grav. 32, 105012.

\item Aasi, J. et al. 2015. Advanced LIGO. Classical and Quantum Gravity 32(7), 074001. 

\item Aasi, J. et al. 2015. Searching for stochastic gravitational waves using data from the two colocated LIGO Hanford detectors. Physical Review Letters 91, 022003. 

\item Aasi, J. et al. 2014. Improved upper limits on the stochastic gravitational-wave background from 2009-2010 LIGO and Virgo data. Physical Review Letters 113(23),  231101. 

\item Aartsen, M. et al. 2014. Multimessenger search for sources of gravitational waves and high-energy neutrinos: Initial results for LIGO-Virgo and IceCube. Physical Review D 90(10), 102002.

\item Aasi, J. et al. 2014. First all-sky search for continuous gravitational waves from unknown sources in binary systems. Physical Review D  90(6), 062010. 

\item Aasi, J. et al. 2014. Search for gravitational waves associated with gamma-ray bursts detected by the interplanetary network. Physical Review Letters 113(1), 011102. 

\item Aasi, J. et al. 2014. The NINJA-2 project: detecting and characterizing gravitational waveforms modelled using numerical binary black hole simulations. Classical and Quantum Gravity 31(11),  115004.  arXiv:1401.0939

\item Aasi, J. et al. 2014. Constraints on cosmic strings from the LIGO-Virgo gravitational-wave detectors. Physical Review Letters 112(13), 131101. 

\item Aasi, J. et al. 2014. Application of a Hough search for continuous gravitational waves on data from the fifth LIGO science run. Classical and Quantum Gravity 31(8), 085014. 

\item Aasi, J. et al. 2014. First searches for optical counterparts to gravitational-wave candidate events. Astrophysical Journal Supplement Series 211(1), 7. 

\item  Aasi, J. et al.  2014.  {(\it as internal reviewer)}  Search for gravitational radiation from intermediate mass black hole binaries in data from the second LIGO-Virgo joint science run.  Physical Review Letters 89, 122003.  arXiv:1404.2199

\item Aasi, J. et al. 2014. Implementation of an {\it F}-statistic all-sky search for continuous gravitational waves in Virgo VSR1 data. Classical and Quantum Gravity 31(16), 165014.

\item Aasi, J. et al. 2013. Search for long-lived gravitational-wave transients coincident with long gamma-ray bursts. Physical Review D 88(12), 122004. 

\item Aasi, J. et al. 2013. Directed search for continuous gravitational waves from the Galactic center. Physical Review D 88(10), 102002. 

\item Aasi, J. et al. 2013. Parameter estimation for compact binary coalescence signals with the first generation gravitational-wave detector network. Physical Review D 88(6), 062001.  arXiv:1304.1775

\item Adri\'{a}n-Mart\'{i}nez, S. et al. 2013. A first search for coincident gravitational waves and high energy neutrinos using LIGO, Virgo and ANTARES data from 2007. Journal of Cosmology and Astroparticle Physics 2013(6), 008. 

\item Aasi, J. et al. 2013. Einstein@Home all-sky search for periodic gravitational waves in LIGO S5 data. Physical Review D  87(4), 042001. 

\item Aasi, J. et al. 2013. Enhanced sensitivity of the LIGO gravitational wave detector by using squeezed states of light. Nature Photonics 7(8),  613-619.

\item Aasi, J. et al. 2013. Search for gravitational waves from binary black hole inspiral, merger, and ringdown in LIGO-Virgo data from 2009-2010. Physical Review D  87(2), 022002. 

\item Abadie, J. et al. 2012. Upper limits on a stochastic gravitational-wave background using LIGO and Virgo interferometers at 600-1000 Hz. Physical Review D  85(12), 122001. 

\item Abadie, J. et al. 2012. All-sky search for gravitational-wave bursts in the second joint LIGO-Virgo run. Physical Review D 85(12), 122007. 

\item Abadie, J. et al. 2012. First low-latency LIGO+Virgo search for binary inspirals and their electromagnetic counterparts. Astronomy \& Astrophysics 541, A155. 

\item Abadie, J. et al. 2012. Search for gravitational waves from intermediate mass binary black holes. Physical Review D, 85(10), 102004. 

\item Abadie, J. et al. 2012. Implementation and testing of the first prompt search for gravitational wave transients with electromagnetic counterparts. Astronomy \& Astrophysics 539, A124.

\item Abadie, J. et al. 2012. All-sky search for periodic gravitational waves in the full S5 LIGO data. Physical Review D 85(2), 022001.

\item Abadie, J. et al. 2012. Implications for the origin of GRB 051103 from LIGO observations. Astrophysical Journal 755(1), 2. 

\item Abadie, J. et al. 2012. Search for gravitational waves from low mass compact binary coalescence in LIGO's sixth science run and Virgo's science runs 2 and 3. Physical Review D 85(8), 082002.

\item Aasi, J. et al. 2012. The characterization of Virgo data and its impact on gravitational-wave searches. Classical and Quantum Gravity 29(15), 155002.

\item Evans, P. et al. 2012. SWIFT follow up observations of candidate gravitational-wave transient events. The Astrophysical Journal Supplement Series 203(2), 28. 

\item Abadie, J. et al. 2012. Search for gravitational waves associated with gamma-ray bursts during LIGO science run 6 and Virgo science runs 2 and 3. Astrophysical Journal 760(1), 12. 

\item Abadie, J. et al. 2011. Directional limits on persistent gravitational waves using LIGO S5 science data. Physical Review Letters 107(27),  271102. 

\item  Abadie, J. et al.  2011. {\it (as internal reviewer)}  Search for gravitational waves from binary black hole inspiral, merger and ringdown.  Phys.~Rev.~D 83:122005. arXiv:1102.3781

\item Abadie, J. et al. 2011. Search for gravitational waves associated with the August 2006 timing glitch of the Vela pulsar. Physical Review D 83(4), 042001.

\item Abadie, J. et al. 2011. Search for gravitational wave bursts from six magnetars. The Astrophysical Journal 734(2), L35.

\item Abadie, J. et al. 2011. Beating the spin-down limit on gravitational wave emission from the Vela pulsar. Astrophysical Journal 737(2),  93. 

\item  Abadie, J. et al. 2010.  {\it (as lead author)}  Predictions for the Rates of Compact Binary Coalescences Observable by Ground-based Gravitational-wave Detectors.  Class.~Quant.~Grav.~27, 173001. arXiv:1003.2480  

\item Abbott, B. et al. 2010. Search for gravitational-wave bursts associated with gamma-ray bursts using data from LIGO science run 5 and Virgo science run 1. Astrophysical Journal 715(2), 1438-1452. 

\item Abbott, B. et al. 2010. Searches for gravitational waves from known pulsars with science run 5 LIGO data. Astrophysical Journal 713(1), 671-685. 

\item Abadie, J. et al. 2010. All-sky search for gravitational-wave bursts in the first joint LIGO-GEO-Virgo run. Physical Review D 81(10), 102001. 

\item Abadie, J. et al. 2010. First search for gravitational waves from the youngest known neutron star. Astrophysical Journal 722(2), 1504-1513. 

\item Abadie, J. et al. 2010. Calibration of the LIGO gravitational wave detectors in the fifth science run. Nuclear Instruments and Methods in Physics Research Section A, 624(1), 223-240. 

\item Abadie, J. et al. 2010. Search for gravitational waves from compact binary coalescence in LIGO and Virgo data from S5 and VSR1. Physical Review D 82(10), 102001. 

\item Abadie, J. et al. 2010. Search for gravitational-wave inspiral signals associated with short gamma-ray bursts during LIGO's fifth and Virgo's first science run. The Astrophysical Journal 715(2), 1453-1461. 

\item Abbott, B. et al. 2009. Search for gravitational wave ringdowns from perturbed black holes in LIGO S4 data. Physical Review D 80(6), 062001. 

\item Abbott, B. et al. 2009. First LIGO search for gravitational wave bursts from cosmic (super)strings. Physical Review D 80(6), 062002. 

\item Abbott, B. et al. 2009. LIGO: the laser interferometer gravitational-wave observatory. Reports on Progress in Physics 72(7), 076901. 

\item Abbott, B. et al. 2009. Einstein@Home search for periodic gravitational waves in early S5 LIGO data. Physical Review D 80(4), 042003. 

\item Abbott, B. et al. 2009. Search for gravitational waves from low mass binary coalescences in the first year of LIGO's S5 data. Physical Review D 79(12), 122001. 

\item Abbott, B. et al. 2009. Stacked Search for Gravitational Waves from the 2006 SGR 1900+14 Storm. Astrophysical Journal 701(2), L68-L74. 

\item Abbott, B. et al. 2009. Search for high frequency gravitational-wave bursts in the first calendar year of LIGO's fifth science run. Physical Review D 80(10),  102002. 

\item Abbott, B. et al. 2009. All-Sky LIGO Search for Periodic Gravitational Waves in the Early Fifth-Science-Run Data. Physical Review Letters 102(11), 111102. 

\item Abbott, B. et al. 2009. An upper limit on the stochastic gravitational-wave background of cosmological origin. Nature 460(7258), 990-994. 

\item Abbott, B. et al. 2009. Search for gravitational-wave bursts in the first year of the fifth LIGO science run. Physical Review D 80(10), 102001. 

\item Abbott, B. et al. 2009. Observation of a kilogram-scale oscillator near its quantum ground state. New Journal of Physics 11(7), 073032.

\item Abbott, B. et al. 2009. Search for gravitational waves from low mass compact binary coalescence in 186 days of LIGO's fifth science run. Physical Review D 80(4), 047101. 

\end{enumerate}

\section{\sc Recently submitted publications, conference proceedings}

\begin{enumerate}

\item A. Levan et al. 2022.  A long-duration gamma-ray burst of dynamical origin from the nucleus of an ancient galaxy.

\item C.~Kobayashi, I.~Mandel, K.~Belczynski, S.~Goriely, T.~Janka, O.~Just, A.~Ruiter, D.~Vanbeveren, M.~Kruckow, M.~Briel, J.~Eldridge, E.~Stanway.  2022.  Can neutron star mergers alone explain the r-process enrichment of the Milky Way?    arXiv:2211.04964

\item M.~Lau, M.~Cantiello, A.~Jermyn, M.~MacLeod, I.~Mandel, D.~Price.  2022.  Hot Jupiter engulfment by a red giant in 3D hydrodynamics.  arXiv:2210.15848

\item T.~N.~O'Doherty, A.~Bahramian, J.~C.~A.~Miller-Jones, A.~J.~Goodwin, I.~Mandel, R.~Willcox, P.~Atri, J.~Strader.  2022.  An Observational Kick Distribution for 145 Neutron Stars in Binaries.  

\item I.~Agudo et al. 2022.  Panning for gold, but finding helium: discovery of the ultra-stripped supernova SN2019wxt from gravitational-wave follow-up observations.  arXiv:2208.09000

\item T.~Wilson, A.~Casey, I.~Mandel, E.~Bellinger, G.~Davies.  2021.  What can rotational splittings of low-luminosity subgiants actually tell us about the rotation profile?  arXiv:2111.10953

\item V.~Kalogera et al.  2021. The Next Generation Global Gravitational Wave Observatory: The Science Book.  arXiv:2111.06990

\item I.~Mandel.  2021.  An accurate analytical fit to the gravitational-wave inspiral duration for eccentric binaries. RNAAS 5, 223. arXiv:2110.09254

\item L.~Urias, T.~J.~Maccarone, V.~Antoniou, I.~Mandel, S.~Vinciguerra.  2021.  Candidate Type II Be X-Ray Binary Outbursts in NGC 6744.  Res.~Notes AAS 5, 209.

\item A.~Casey, I.~Mandel, P.~Ray. 2021. The impact of the COVID-19 pandemic on academic productivity.  arXiv:2109.06591

\item S.~L.~Schr{\o}der, M.~MacLeod, E.~Ramirez-Ruiz, I.~Mandel, T.~Fragos, A.~Loeb, R.~W.~Everson. 2021. The Evolution of Binaries under the Influence of Radiation-Driven Winds
  from a Stellar Companion.  arXiv:2107.09675

\item ``I 'heart' Pluto'' -- we 'heart' it, mostly'' (book review).  I.~Mandel.  2020.  Crystallography Reviews

\item M.~Bailes et al. 2019. Ground-Based Gravitational-Wave Astronomy in Australia: 2019 White Paper. arXiv:1912.06305

\item D.~Liptai, D.~J. Price, I.~Mandel, G.~Lodato.  2019.  Disc formation from tidal disruption of stars on eccentric orbits by Kerr black holes using GRSPH.  arXiv:1910.10154

\item P.~F.~Michelson, R.~L.~Byer, S.~Buchman, I.~Mandel, J.~Lipa, S.~Saraf.  2019. MFB: A Mid-Frequency-Band Space Gravitational Wave Observer for the 2020 Decade (Decadal Instrument White Paper). arXiv:1908.02861 

\item V.~Kalogera, C.~P.~L.~Berry, M.~Colpi, S.~Fairhurst, S.~Justham, I.~Mandel, A.~Mangiagli, M.~Mapelli, C.~Mills, B.~S.~Sathyaprakash, R.~Schneider, T.~Tauris, R.~Valiante. 2019.  Deeper, Wider, Sharper: Next-Generation Ground-Based Gravitational-Wave Observations of Binary Black Holes (Decadal White Paper).  arXiv:1903.09220 

\item J.~W.~Barrett, I.~Mandel, C.~J.~Neijssel, S.~Stevenson, A.~Vigna-G\'{o}mez. 2017. Exploring the Parameter Space of Compact Binary Population Synthesis. IAU Symposium 325, 46. arXiv:1704.03781 

\item I.~Mandel.  2017. The astrophysics of LIGO gravitational-wave observations, in {\it 11th INTEGRAL conference proceedings}, edited by Ed van den Heuvel, \url{https://pos.sissa.it/285/002}

\item C.~P.~L.~Berry et al.  2016.  Early Advanced LIGO binary neutron-star sky localization and parameter estimation.  Journal of Physics: Conference Series; 76(1):012031(4).  arXiv:1606.01095

\item W.~M.~Farr, I.~Mandel, C.~Aldridge, K.~Stroud. 2015. The Occurrence of Earth-Like Planets Around Other Stars.  arXiv:1412.4849
  
\item M.~Hendry et al. 2014. Education and public outreach on gravitational-wave astronomy.  General Relativity and Gravitation, 46:1764.

\item R.~M.~Shannon et al. 2014. Summary of session C1: pulsar timing arrays. General Relativity and Gravitation, 46:1765.

\item M.~Branchesi et al. 2014.  C7 multi-messenger astronomy of GW sources.
General Relativity and Gravitation, 46:1771.

\item I.~Mandel et al.  2014.  Relativistic astrophysics at GR20.  General Relativity and Gravitation, 46:1688.

\item D.~Buskulic and I.~Mandel.  2013.  LIGO and Virgo Gravitational-wave Detectors and Their Science Reach.  Acta Physica Polonica B 44, 12, 2413.

\item I.~Mandel, L.~Z.~Kelley, E.~Ramirez-Ruiz. 2011. Towards improving the prospects for coordinated gravitational-wave and electromagnetic observations. arXiv:1111.0005
 
\item I.~Mandel, R.~O'Shaughnessy, V.~Kalogera.  2010.  Unravelling Binary Evolution from Gravitational-Wave Signals and Source Statistics. arXiv:1001.2583 

\item I.~Mandel, J.~R.~Gair, M.~C.~Miller.  2010.  Detecting coalescences of intermediate-mass black holes in globular clusters with the Einstein Telescope.  arXiv:0912.4925  

\item J.~R.~Gair, I.~Mandel, L.~Wen. 2008.  Time-Frequency Analysis of 
Extreme-Mass-Ratio Inspiral Signals in Mock LISA Data.  
J. Phys.: Conf. Ser. 122 012037.  arXiv:0710.5250

\item I.~Mandel. 2007. Spin Distribution Following Minor Mergers and the 
Effect of Spin on the Detection Range for Low-Mass-Ratio Inspirals. 
arXiv:0707.0711

\item A.~Silbergleit, I.~Nemenman, I.~Mandel. 2003. On the Interaction of 
Point Charges in an Arbitrary Domain.  J.~Tech.~Phys.~ 48 (2), 146-151.  
arXiv:physics/0105052

\end{enumerate}

\newpage
\section{\sc Selected Presentations}
9/2006 	 Talk  at LISA Astro-GR@AEI, Golm, Germany. 	 ``Using EMRIs to probe bumpy black-hole spacetimes''

1/2007 	Seminar at Leyden, Netherlands. 	``Gravitational waves from intermediate-mass-ratio inspirals in ground-based detectors''

4/2007 	Talk 	at APS, Jacksonville, FL. 	``Intermediate-mass-ratio inspirals into intermediate-mass black holes''

7/2007 	Talk 	at Amaldi 7, Sydney, Australia. 	``Time-Frequency Analysis of Extreme-Mass-Ratio Inspiral Signals: Mock LISA Data Challenge, Round 2''

10/2007 	Astrophysics Seminar at Northwestern University, Evanston, IL. 	``Testing the No-Hair Theorem with Gravitational-Wave Observations''

11/2007 	Talk 	at 17th Midwest Relativity Meeting, St Louis, MO. 	``Black-Hole Spins Following Minor Mergers''

12/2007 	Talk 	at GWDAW-12, Boston, MA. 	``Extracting Extreme Mass Ratio Inspiral Parameters via Time-Frequency Methods''

2/2008 	Seminar at Center for Gravitation and Cosmology, University of Wisconsin-Milwaukee. 	``Ground-based detection of gravitational waves from intermediate-mass-ratio inspirals''

3/2008 	Talk at 24th Pacific-Coast Gravity Meeting , Santa Barbara, CA. 	``Extracting Extreme Mass Ratio Inspiral Parameters via Time-Frequency Methods''

6/2008 	Talk at 7th LISA Symposium, Barcelona, Spain. 	``Can we detect IMRIs?''

9/2008 	Short seminar  at Institute of Astronomy, Cambridge, UK. 	``Ground-based detection of gravitational waves from intermediate-mass-ratio inspirals'' 

10/2008 	Talk 	at 18th Midwest Relativity Meeting, Notre Dame, IN. 	``Can we detect intermediate-mass-ratio inspirals?''

12/2008 	Seminar at HEP/Astrophysics Seminar, Purdue, West Lafayette, IN. 	``Prospects in Gravitational-Wave Astronomy'' 

%1/2009 	Poster 	GWDAW-13 San Juan, Puerto Rico 	``Can we detect intermediate-mass-ratio inspirals?''

2/2009 	Seminar 	Southampton, UK 	``Compact binaries as sources for ground-based gravitational-wave detectors''

3/2009 	Seminar 	KITP, UC Santa Barbara 	``Compact Binaries, Intermediate-Mass-Ratio Inspirals, and Other Prospects in Graviational-Wave Astronomy''

4/2009 	Talk 	at IMBH Workshop, UC Irvine. 	``Gravitational Waves from Binary Systems Containing Intermediate-Mass Black Holes'' 

6/2009	Talk at Amaldi-8 conference, Columbia University. ``Compact binaries as sources for ground-based gravitational-wave detectors''

7/2009	Invited Talk at NRDA-2 at AEI, Golm, Germany. ``Predictions for Detectable Coalescences of Compact Binaries Including Black Holes''

7/2009	Talk at Marcel Grossmann 12 in Paris, France.  ``Probing Light Seeds of Massive Black Holes with Gravitational Waves'' 

7/2009	Invited talk at MG-12, Paris.  ``Unravelling Binary Evolution from Gravitational-Wave Signals and Source Statistics''

10/2009	Invited talk at the CfA, Cambridge, MA.  ``Gravitational Waves from Binaries''

1/2010	Talk at NSF Fellows' Symposium, AAS, Washington, DC. ``Prospects in Gravitational-Wave Astronomy''

1/2010	Poster at GWDAW, Rome.  ``Parameter estimation on gravitational waves from multiple coalescing binaries''

2/2010	Talk at Aspen Winter School on Formation and Evolution of Black Holes, Aspen, CO.  ``Extracting the distribution of black-hole parameters from gravitational-wave observations''

3/2010	Seminar, University of California, Santa Cruz. ``Prospects in Gravitational-Wave Astronomy''

5/2010	Seminar, Northwestern University. ``Prospects in Gravitational-Wave Astronomy''

5/2010	Seminar, Rochester Institute of Technology. ``Prospects in Gravitational-Wave Astronomy''

6/2010	Talk at NRDA at the Perimeter Institute, Waterloo, Canada. ``Bayesian Inference on Numerical Injections''

6/2010	Poster at LISA Symposium, Stanford, CA.  ``Extracting the distribution of black-hole parameters from gravitational-wave observations''

7/2010	Invited talk at GR-19, Mexico City.  ``The Mock LISA Data Challenges''

7/2010	Seminar, University of Birmingham, UK.  ``GWAstrophysics''

7/2010	Talk, COSPAR-2010, Bremen.  ``Testing General Relativity with Gravitational Waves from Extreme Mass Ratio Inspirals''

1/2011 	Talk at NSF Fellows Symposium, Seattle. 	``Gravitational waves from compact binaries: Status and prospects''

1/2011 	Talk at AAS, Seattle. 	``Searching For Gravitational-wave Signals From Compact Binaries With LIGO And Virgo''

1/2011 	Talk at GWPAW, Milwaukee, WI.	 ``The Distribution of Coalescing Compact Binaries in the Local Universe: Prospects for Gravitational-Wave Observations''

2/2011 	Seminar, West Virginia University, Morgantown, WV. 	``Markov Chain Monte Carlo techniques for parameter estimation''

2/2011 	Seminar, University of Florida, Gainesville, FL. 	``GWastrophysics with compact binaries''

3/2011 	Talk at Evolution of Compact Binaries, Valparaiso, Chile. 	``Compact Binary Coalescences: Connections to Gravitational-Wave Astronomy'' 

4/2011 	Talk 	at MKI Postdoc Symposium, MIT. 	``Gravitational-wave Astrophysics with Compact Binaries''

5/2011 	Seminar 	at Princeton University.	``GWastrophysics''

5/2011 	Seminar 	at University of Maryland, College Park. 	``GW astrophysics with compact binaries''

5/2011 	Talk at Aspen Center for Physics. 	``Why compact binaries are not boring''

9/2011	Poster at New Horizons in Time Domain Astronomy, Oxford, UK.  ``Electromagnetic transients as triggers in searches for gravitational waves from compact binary inspirals''

10/2011	Invited talk at LOFT Science Meeting, Amsterdam. ``Gravitational wave observatories and LOFT''

11/2011	Seminar at Warsaw University, Poland. ``Gravitational-wave Astrophysics of Compact Binaries''

12/2011	Seminar at Cardiff University. 	``GWastrophysics of Compact Binaries''

12/2011	Invited talk at LOFT and the Variable X-ray Sky, RAS, London. ``LOFT and the EM counterparts to gravitational wave sources''

12/2011	Talk at Gravitation, Astrophysics and Cosmology conference, Quy Nhon, Vietnam.  ``Astrophysics, Cosmology and Fundamental Physics with ground-based gravitational-wave detectors''

3/2012 	Invited talk at April APS Meeting, Atlanta, GA. 	``Intermediate-mass black holes: A theoretical perspective''

4/2012 	Talk 	at April APS Meeting, Atlanta, GA.  ``What Waveforms do Data Analysts Want?, or the dangers of systematic errors in parameter estimation''

5/2012 	Invited talk at Sackler conference: testing GR with astrophysical systems, Cambridge, MA. 	``Testing GR with Binary Coalescence Events''

6/2012 	Talk 	at GWPAW, Hannover.  ``What Waveforms do Data Analysts Want?, or the dangers of systematic errors in parameter estimation''

6/2012 	Invited talk at Exploring New Horizons with Gravitational Waves, Hannover.   ``Astrophysical Sources and Parameter Estimation''

7/2012	Invited course at Nijmegen Astroparticle summer school.  ``Gravitational Wave Astrophysics''

9/2012	Led discussion on parameter estimation for advanced detectors at KITP workshop, Santa Barbara

10/2012	Seminar at University of Nottingham.  ``GWastrophysics of Compact Binaries'' 

10/2012	Invited talk at Russian Young Scientists Conference on Physics and Astronomy, Saint Petersburg (in Russian)

12/2012	Talk at Einstein Telescope Meeting, Hannover.  ``Challenges for source parameter estimation with the Einstein Telescope''

1/2013 	Invited lectures on gravitational-wave astrophysics and parameter estimation at the Chris Engelbrecht Gravitational-Wave Summer School, South Africa

2/2013	Seminars at Cambridge University and University of Warwick.  ``GWastrophysics of Compact Binaries'' 

2/2013	Invited overview talk at UK-India LIGO meeting.  ``Science Exploitation of advanced GW detectors: a provocation''

3/2013	Invited LIGO Academic Advisory Council lecture on ``Astrophysics of binary black holes''

4/2013 	Seminar at Hebrew University of Jerusalem.  ``GWastrophysics with Compact Binaries'' 

6/2013	Seminars at Guelph University and Perimeter Institute.  ``Exploring (astro)physics with gravitational waves''

6/2013	Invited talk at First UK LOFT Science meeting, London.  ``Gravitational Waves 
and LOFT''

7/2013	Invited lecture on LIGO and compact-binary astrophysics, Cracow School of Theoretical Physics, Zakopane, Poland

7/2013	Talk at GR-20, Warsaw.  ``BiG Waves: a different kind of gravitational-wave summer school''
 
8/2013	Invited lectures on gravitational-wave astrophysics at Beijing Normal University gravitational-wave summer school
 
9/2013 	Talk at Numerical Relativity and Data Analysis workshop, Mallorca. ``Waveform accuracy requirements for parameter estimation''

10/2013 	Seminar at Imperial College, London. ``LIGO and GWastrophysics of compact binaries''

12/2013   Talk at Gravitational-Wave Physics and Astronomy Workshop (GWPAW), Pune, India.  ``Parameter estimation, Fisher matrices, and abruptly terminating waveforms''.  Invited panelist for discussion of multi-messenger astronomy

1,2/2014	Seminars at University of Kwa-Zulu Natal, University of Western Cape, and Stellenbosch University, South Africa, as part of a long-term National Institute of Theoretical Physics fellowship

3/2014	Seminars at Newcastle University, Florida Atlantic University

3/2014	Talk at ``Stellar Tango in the Rockies'' workshop, Lake Louise, Canada.  ``What can gravitational waves teach us about compact binary evolution?''

7/2014	Solicited talk at University of Washington INT workshop on ``Binary Neutron Star Coalescence as a Fundamental Physics Laboratory''

8/2014 	Seminar at University of California, San Diego.   ``Gravitational-wave astrophysics of compact binaries''

10/2014	Seminar at University of Southampton.  ``Adventures in Astrostatistics''

10/2014	Seminar (``Adventures in Astrostatistics'') and Tutorial (``All you wanted to know about Gravitational Wave Astrophysics and were afraid to ask'') at University of Amsterdam.  

2/2015	Astronomy seminars at Melbourne and Monash Universities

4/2015 	Physics colloquium at Monash University; astronomy seminar at Swinburne University:  ``Beautiful Binaries''

5/2015	Invited talk at Cardiff Black Hole Workshop: ``Black-hole mass measurements''

6/2015	Invited talk at ``GRG: A centennial perspective'', Penn State: ``Overview of compact-binary merger rate predictions''

7/2015	Invited talks at Marcel Grossmann 14, Rome: ``Intermediate-mass black holes: A theoretical perspective'' and ``Prospects for BNS Observations with Advanced GW Detectors''

7/2015	Talk at Alpine Cosmology Workshop: ``Cosmology with gravitational waves''

11/2015	Invited talk, Jerusalem Tidal Disruption Workshop

2/2016 	Invited talk, Aspen conference on Dynamics and accretion at the Galactic Center: ``Tidal Disruptions of Binary Stars''

2/2016	Physics department colloquium, University of Florida;  Astronomy seminar, Durham University: ``Gravitational-wave astrophysics: the future is now''

3/2016	Invited talk at Workshop on Sampling in higher dimensions, Edinburgh: ``Enhancing sampling efficiency: Inference on binary black holes with gravitational waves''

5/2016 	Invited talk at The First Observations of a Binary Black Hole Merger, Hannover: ``GW150914: Astrophysical implications of the discovery''

7/2016	Talk, Binary Stars in Cambridge, Cambridge, UK: ``Massive Binary Paleontology with Gravitational Waves''

8/2016 	Talk, GRavitational-wave Astronomy Meeting in PAris (GRAMPA): ``Gravitational-wave Palaeontology''

9/2016 	Colloquia, Leiden University and Center for Astrophysics, Harvard University: ``Gravitational-wave astrophysics: The future is now'' {\it https://www.youtube.com/watch?v=xXH-NBdhzk0}

10/2016 	Invited talk at INTEGRAL conference, Amsterdam: ``The astrophysics of LIGO gravitational-wave observations''

10/2016 	Invited talk at Gravitational Waves \& Cosmology meeting, DESY, Hamburg:  ``Astrophysical consequences of the LIGO discovery''

10/2016 	Invited lectures at Marie Curie GraWIToN school, Rome

11/2016 	Colloquium at the Geneva Observatory; Informal seminar at the Copernicus Center, Warsaw 

12/2016 	Colloquium at University of Nottingham

1/2017 	Colloquium at Cardiff University

2/2017	Invited conference summary lecture, Aspen Center for Physics winter conference on ``The Dawning Era of Gravitational-Wave Astrophysics''

2/2017	Colloquium at Lund University, Sweden 

5/2017	Colloquium at Albert Einstein Institute (Max Planck Institute), Golm bei Potsdam

6/2017 	Colloquium at University of Manchester

9/2017	Workshop summary talk at Lorentz Center, Leiden: ``And then there was light: electromagnetic counterparts of binary black hole mergers''

9/2017	Invited talk at ``Piercing the Sphere of Influence'' conference, Cambridge: ``Stellar binaries: tidal separations, mergers, and disruptions''

9/2017	Plenary talk at DESY theory workshop ``Fundamental physics in the cosmos'', Hamburg: ``Astrophysical sources of GWs and future prospects for their detection''

10/2017	Seminar at Oxford University

11/2017 	Invited discussion lead at Center for Computational Astrophysics (New York) special workshop on binary neutron stars

12/2017 	Seminar at Keele University

2/2018  	Seminars at University of California, Berkeley, and University of California, Santa Cruz, ``The future of gravitational-wave astronomy''

4/2018	Seminar at St Petersburg Academic University (in Russian)

4/2018	Invited talk at ``UQ for inverse problems in complex systems'', Isaac Newton Institute, Cambridge: ``Studying black holes with gravitational waves: Why GW astronomy needs you!''

5/2018 Invited talk at Sackler conference on Gravitational-wave Astrophysics: ``Formation of merging black holes through isolated binary evolution via the common envelope phase'' \url{https://www.youtube.com/watch?v=QhPdlWvWnI0}

7/2018 Les Houches lecture series on astrophysical sources of gravitational waves

8/2018 Plenary talk, IAU General Assembly in Vienna (X-ray binary session), ``Gravitational-wave astrophysics''

8/2018 Lecture at STFC summer school, Belfast

9/2018 Invited presentation at ULX meeting, ISSI, Bern

10/2018 Invited presentation at third-generation gravitational-wave detector science case planning meeting, Golm

2/2019 Seminar at Melbourne University

7/2019 Talk at ASA meeting

8/2019 Invited presentation at International Statistical Institute World Congress (Kuala-Lumpur, Malaysia), ``Adventures in astrostatistics: Black holes and beyond''

8/2019 Talk at Munich Institute for Astro- and Particle Physics program ``Precision gravity: from the LHC to LISA''

10/2019 Seminar at University of Western Australia

1/2020 Invited talk at Max-Planck meeting at Schloss Ringberg, ``The current astrophysical understanding of the progenitors of binary mergers seen by LIGO''

2/20 Talk at ANITA meeting, Canberra

5/20 Colloquium on ``Gravitational-wave astronomy'' at Macquarie University (Sydney, Australia, delivered via zoom)

9/20 Colloquium on  ``Gravitational-wave astronomy'' at TIFR (Mumbai, India, delivered via zoom)

1/2021 Colloquium on ``The promise of gravitational-wave astrophysics'' at UCLA (delivered via zoom)

5/2021 Colloquium at the National Institute of Theoretical Physics, South Africa (delivered via zoom)

6/2021 Colloquium on ``Some recent results in massive binary evolution''  at the Hebrew University of Jerusalem (delivered via zoom)

6/2021 Invited talk on ``Stochastic recipes for compact remnant masses, and natal kicks and their impact on gravitational wave sources'', Gravitational Wave Astrophysics Conference, Hefei, China  (delivered via zoom)

6/2021 Talk on ``Cygnus X-1 as a probe of massive stellar and binary evolution" at the European Astronomical Society meeting (delivered via zoom)

6/2021 Talk on ``Black hole masses (of stellar-mass BHs)'' at Aspen Center for Physics (delivered via zoom)

7/2021 Opening plenary talk, Amaldi 14 ``Accelerating gravitational wave science'', Melbourne, Australia

9/2021 Colloquium at the Kavli Institute for Astronomy and Astrophysics (KIAA), Peking University (delivered via zoom)

10/2021 Astrophysics seminar at the National Centre for Nuclear Research, Warsaw, Poland (via zoom)

11/2021 International Astrostatistics Association and the Astroinformatics \& Astrostatistics Commission of the International Astronomical Union invited seminar (delivered via zoom)

12/2021 LSST Australia workshop, talk on ``Exploring common envelopes with LSST observations of luminous red novae''

1/2022 Lectures at the Saas Fee gravitational waves astrophysics school, Switzerland

3/2022 Colloquium at the University of Melbourne, ``Accretion and drag in stellar binaries'' 

7/2022 Discussion lead, Aspen Center for Physics

9/2022 Colloquium at the University of Queensland

11/2022 Australian National University Colloquium

12/2022	Invited talk at ``Supernovae in the Gravitational Wave Detection Era'', Melbourne, on ``Supernovae and compact-object binaries''


\newpage

\section{\sc Selected Outreach Activities}
11/2008 	LIGO Scientific Collaboration booth at the Society of Physics Students congress at Fermilab

12/2008 	Co-presented {\it The Wonders of the Night Sky: The Life and Death of Stars} at the Theodore Roosevelt Elementary School's science fair in Park Ridge, Illinois

4/2009 	Special event at Dearborn Observatory as part of the {\it 100 Years of Astronomy} celebration of the International Year of Astronomy

5/2009	Science Club at Roosevelt elementary school

5/2009	Judging Meaningful Science Consortium {\it Project Showcase} for Chicago public high school students

7/2009 	 LIGO traveling exhibit at Adler Planetarium. 

12/2009 	Co-presented {\it The Wonders of the Night Sky: The Life and Death of Stars} at the Theodore Roosevelt Elementary School's science fair in Park Ridge, Illinois

3/2010  	Co-organized the presentation of {\it Einstein's Cosmic Messengers}, a multimedia concert by Andrea Centazzo and Michele Vallisneri, at Northwestern University.

2/2011	Taught elementary school children at Anova school in Melrose, MA.

3/2012	Gravitational-wave presentation at the Big Bang Fair, Birmingham National Exhibition Center

2012--2016	Founder and lead organizer of the Birmingham Gravitational-wave summer school (BiG Waves)

6/2012	Presented and led interactive sessions at Physics Experience Week for local high school students

6-8/2012	Supervised a summer student in creating animations of gravitational-wave sights and sounds for public demonstrations

10/2012	Popular physics lecture to Birmingham Humanist society

2012		Edited chapters for Birmingham e-book on gravitational waves, \url{http://www.gwoptics.org/ebook/}
	  
3/2013	Popular article on ``Black Holes in Advanced LIGO: The observational payoff'' (with Ben Farr), \url{http://www.ligo.org/magazine/LIGO-magazine-issue-2.pdf}

3/2014	National Science and Engineering Competition judge

3/2014	Source for March 2014 cover page National Geographic article on black holes,\\ see \url{http://ngm.nationalgeographic.com/2014/03/black-holes/finkel-text}

9/2014	British Science Festival gravitational-wave hands-on exhibit

11/2014	National Science and Engineering Competition online judge

10/2012 --- now 	Member of University of Birmingham HiSPARC team \url{http://www.hisparc.nl/en/}; advisor on data analysis strategies

3/2015	Popular article on ``Where should I apply for grad school?'', LIGO magazine,\\ \url{http://www.ligo.org/magazine/LIGO-magazine-issue-6.pdf\#page=24}

5/2015 	Public talk for Walsall Astronomical Society

1/2016	Public talk for the Association for Science Education annual conference (describing current research to school science teachers)

1/2016 	School talk at Malvern College

1/2016 	School talk at University of Birmingham School

1/2016	Institute of Physics public lecture

1/2016	Public talk as part of the ``Astronomy in the City'' series

2/2016	Source for Nature magazine news article,\\ \url{http://www.nature.com/news/einstein-s-gravitational-waves-found-at-last-1.19361}

2/2016	Interviewed by Aspen public radio

3/2016	Public talk for Liberal Arts and Sciences 

5/2016	``Pint of Science: Night with the stars'' public talk

6/2016	Source for National Geographic news article,\\ \url{http://news.nationalgeographic.com/2016/06/}\\\url{gravitational-waves-stars-ripples-space-time-origins-astronomy}

7/2016	Aspen public physics lecture: ``Singing Binaries: Listening to the Chirps of Black Holes'',\\\url{http://mc.grassrootstv.org/CablecastPublicSite/show/14164?channel=1}

7/2016 	Source for news article\\ \url{http://www.birmingham.ac.uk/university/colleges/eps/news/college/2016/7/Merging-binary-black-holes-formed-through-chemically-homogeneous-evolution.aspx}

9/2016 	In the press for proposing the chemically homogeneous evolution channel with Selma de Mink \url{https://www.quantamagazine.org/colliding-black-holes-tell-new-story-of-stars-20160906/}

12/2016	Talk at Wolverhampton Astronomical Society

1/2017	Talk for school children at Year 9 Big Quiz

3/2017	Invited talk at physics teachers conference, Santa Barbara

4/2017	In the press for Nature Communications article on binary black hole formation: including The Register, %{\it https://www.theregister.co.uk/2017/04/06/how_black_hole_mergers_found_by_ligo_formed},
Physics World\\ \url{http://physicsworld.com/cws/article/news/2017/apr/20/computer-model-helps-explain-how-ligo-s-black-holes-formed}

7/2017	Talk at King Edward's VI Camp Hill School

7/2017	Talk for school children at university residential summer school

7/2017 	Lecture at the Worshipful Company of Scientific Instrument Makers, London

8/2017	Authored article ``Gravitational waves are helping us crack the mystery of how pairs of black holes form'' in The Conversation,\\ 
\url{https://theconversation.com/gravitational-waves-are-helping-us-crack-the-mystery-of-how-pairs-of-black-holes-form-82789}

9/2017	School talk at University of Birmingham School

3/2018	Radio interview on BBC West Midlands

4/2018	Source for Nature News Feature, \url{https://www.nature.com/articles/d41586-018-04157-6}

7/2018 	Source for article in The Independent, \url{https://www.independent.co.uk/life-style/blood-moon-lunar-eclipse-mood-astronomy-astrology-sun-earth-a8464681.html}\\
\url{https://www.birmingham.ac.uk/schools/physics/news/2018/birmingham-professor-counters-myths-surrounding-blood-moon.aspx}

4/2019	Source for Nature News Feature, \url{https://www.nature.com/articles/d41586-019-01064-2}

8/2019	School talk at LIDER Russian School, Melbourne

8/2019 	Source for Quantum Magazine Article, \url{https://www.quantamagazine.org/to-make-two-black-holes-collide-try-three-20190815/}

2/2020	Lecture at Mornington Peninsula Astronomical Society: \url{https://www.youtube.com/watch?v=KY4JduZC4UQ}

2/2020	Co-author (with Leslie Atkins Elliot) of Cosmos Magazine article ``Teaching Physics to Monks''
	
3/2020 	Source for SpaceAustralia Article, \url{http://spaceaustralia.com/news/modelling-evolution-double-neutron-stars?fbclid=IwAR1fnPhS5Wmsh8xe1y_c5QOGMBG23VgF_36U13CzCl2of9Gk3FTk8hPj-Bg}
	  
4/2020	Source for phys.org article \url{https://phys.org/news/2020-03-unravelling-mystery-black-holes-scientists.html}

5/2020	Source for Herald Sun article

5/2020	Live public talk zoom talk ``Singing binaries: listening to the chirps of black holes'' \url{https://www.youtube.com/watch?v=EbdSdWfr_jk}

9/2020	Wrote article ``Gravitational waves: astronomers spot a black hole so massive they weren't sure it could exist'' for The Conversation: \url{https://theconversation.com/gravitational-waves-astronomers-spot-a-black-hole-so-massive-they-werent-sure-it-could-exist-145474} (in top 5 most read Conversation articles of the month)

9/2020	Source for Nature article \url{https://www.nature.com/articles/d41586-020-02524-w}; see also \url{https://www.quantamagazine.org/possible-detection-of-a-black-hole-so-big-it-should-not-exist-20190828/}

9/2020	Interview, South African radio

10/2020	Source for Science article \url{https://www.sciencemag.org/news/2020/10/famous-shadow-black-hole-provides-novel-test-new-theories-gravity}
	  
12/2020	Australian Institute of Physics Victoria Nobel Lecture

1/2021	Source for ``The Ripple Effect'' article in The Mercury

1/2021 	Subject of ``Using 100-million-year-old fossils and gravitational-wave science to predict Earth's future climate'', e.g., \url{https://phys.org/news/2021-01-million-year-old-fossils-gravitational-wave-science-earth.html}

1/2021	Subject of ``Past (more) perfect'', Cosmos magazine, \url{https://cosmosmagazine.com/history/palaeontology/past-more-perfect}

2/2021 	Public lecture for the Astronomical Society of Victoria (over zoom)

2/2021 	Co-wrote article ``21 times the Sun's mass: heaviest stellar black hole in the Milky Way is more massive than we thought'' for The Conversation: \url{https://theconversation.com/21-times-the-suns-mass-heaviest-stellar-black-hole-in-the-milky-way-is-more-massive-than-we-thought-155484}

2/2021	Source for or mentioned in multiple other articles related to Cygnus X-1 observations, e.g., New York Times: \url{https://www.nytimes.com/2021/02/18/science/cygnus-black-hole-astronomy.html}
	  
9/2021	Public lecture ``New windows on the universe: The discovery of gravitational waves'': \url{https://www.youtube.com/watch?v=ZhR5bJlfp54}

1/2022	Keynote presentation for the National Youth Science Forum

1/2022	Article in ``All about space'' magazine
	  
2/2022  	Source for Nature article \url{https://www.nature.com/articles/d41586-022-00346-6}

2/2022	Astronomical Society of Australia Early Career Researcher workshop keynote presentation ``How to apply for an academic job'' 

5/2022	Source for Daily Express article on Event Horizon Telescope imaging of Sgr A* \url{https://www.express.co.uk/news/science/1609298/black-hole-pictured-milky-way-event-horizon-telescope-sagittarius-a-einstein-law}

5/2022	Talk for students at Melbourne High School
	  
\end{resume}
\end{document}




